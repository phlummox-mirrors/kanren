\documentstyle{jfp}
\RequirePackage{alltt}

% added for proper hyphenation:
\righthyphenmin=2 
\lefthyphenmin=3 

\newcommand{\comment}[1]{}

%% For verbatim displays use the {eopltt} environment, which is like
%% {alltt}, except that it uses the \codefontsize font. 
%% {alltt} is like verbatim
%% except that \ { and } work normally.  This allows the most normal use of
%% tex macros inside code segments.  Be glad we're not using Java or C.
%% Or if macros aren't used in the next and there are literal backslashes,
%% use {eoplverbatim}
\def\myrule{\raisebox{3.4ex}[1.5ex][.75ex]{\rule{126mm}{.01mm}}{\vspace{-3ex}}\\}
%\def\myrule{\raisebox{6ex}{\rule{126mm}{.01mm}}\\}
\def\codefontsize{\fontsize{9}{10}\selectfont\tt} 
\def\inlinecode#1{\hbox{\codefontsize #1\/}}
\long\def\inlinecodebreak#1{{\codefontsize #1\/}}
\newenvironment{verbatimcode}%
  {\endgraf\codefontsize\verbatim}%
  {\endverbatim}
\newenvironment{ttcode}
  {\codefontsize\begin{alltt}}
  {\end{alltt}}
\def\Di{$D_i$}
\def\Ds{$D_s$}
% From Steve
%\Rule{name}{antecedents}{conclusions}{sideconditions?}
\newcommand{\Rule}[4]{
\makebox{{\rm #1}
$\displaystyle
\frac{\begin{array}{l}#2\\\end{array}}
{\begin{array}{l}#3\\\end{array}}$
 #4}}


%--------------------------------------------------------
\begin{document}
\bibliographystyle{jfp}

\title{A ``Poor Man's'' Logic System with First-Class Relations
(Draft: July 2, 2003)} \author[Friedman]{Daniel
P. Friedman \thanks{...}
%\thanks{...}
\\ Computer Science Department, Indiana University 
\\ Bloomington, IN 47405, USA
}
\maketitle[t]

\section{Introduction}
We present an implementation of an embedding in R$^5$
Scheme of a logic system with first-class relations
that we believe interacts smoothly with its host.  How can we have
logic programming and first-class relations and lose nothing of
Scheme?  We make the control structure of logic programming explicit,
so that Scheme programs work with it.  This requires just a small
handful of operations that comprises its interface.

In this introduction we skim over the topics that we cover.  We ask
for the reader's patience when some term or idea is mentioned without
a full explanation.  Keep in mind that we are not only explaining how
to write logic programs, but also how to embed them in Scheme.  All
aspects of logic programming mentioned in this introduction are
described later in the paper.  We do assume, however, a reading
knowledge of Scheme and some familiarity with macros.

Logic systems such as Prolog have been around for quite a while and much is
understood about them.  By the way that we have implemented our logic
system, we have been able to include first-class relations.  Furthermore,
we make them extensible, which means that we give ourselves the power
of sharing subrelations.

In order to make this approach work, four properties that are not
normally associated with the discussion of logic programming had to
become apparent.  The first property is that most of the time we
needn't think about the relations as a monolithic global relation.  In
fact, in most circumstances, the programmer knows exactly which
relation to home in on and we are going to require that knowledge of
the programmer.\footnote{This restriction also appears in Silvija
Seres and Michael Spivey, ``Embedding Prolog in Haskell''. In
\emph{Haskell Workshop}, Paris, France, September, 1999.}  For
example, the relation that infers a type for a programming language
differs significantly from the relation that determines how to
concatenate two lists, and both of these differ considerably from the
relation that determines whether two people are related by the fact
that one of them is the father of someone and the other is the child
of that same someone.  Being able to separate these three relations
makes possible a smooth integration with Scheme.

The second property is that there should be no distinction between a
Scheme function and a logic relation.  This has allowed for the
removal of a dispatch, leading to an implementation with
\emph{substitution composition} being the only place where recursion
appears.  This is not quite accurate for three reasons.  First, if
only one answer is produced, then no recursion is needed, but some
mechanism must be in place to handle no answer or ask for more answers
(We use streams.).  Second, in order to display answers that contain
the same logic variable more than once, we need a recursive copier
that keeps track of the logic variables.  But, this is for passing
results out to arbitrary Scheme functions.  Finally, in the last
section we introduce pairs (lists) as valid terms, requiring the
revision of three functions, each of which becomes recursive.

The third property is that gratuitous lists should never be built.
This matters because we treat relations as finite-arity functions and
thus we can consider the actual arguments separately instead of joined
together as a list.  The implications of this decision are far
reaching, since it allows us to avoid many applications of
substitutions.  Where there would be two lists to unify, we now have
two sequences of arguments, which means that we can avoid substituting
in each \texttt{cdr} in each list. (See the redefinition of
\texttt{unify*} in the last section.)  Also, if our raw data does not
contain lists, then our unifier does not need to know about them.

The fourth property is that since relations are first-class, they
should support a relation extension operation.  This operation takes
an arbitrary number of relations, all with the same signature.  We
start the development in a more conventional way treating a relation
as a single rule.  Then we present relation extension.  With it, we
get the ability to share relations.  We use relation extension
throughout, but we show some of the real power of relation extension
later when we interactively build a type inference system with it.

This tutorial has been written so that most definitions come before
their usage.  This has the disadvantage that occasionally the code for
the implementation appears before it is explained.  It has the
advantage, however, that most lines of code can be accounted for while
sitting at a terminal.  This does not necessarily make learning the
material easier, but it does allow the reader to know what parts
belong in what order.  Skimming the tutorial, however, is mostly
discouraged, since some functions and relations are defined multiple
times and their order of definition matters!

There are six additional sections.  First, we include preliminaries, much
of which should be familiar to most readers.  We introduce some useful
syntactic abstractions as well as some standard functions for dealing with
\emph{logic} variables, substitutions, and unification.  In section 3 we
implement and demonstrate the unsullied logic system.  In section 4 we
explore sullied operators, such as the cut operator, and their
implementation. Next, we utilize some recursive relations and include a
comparison with an embedding in Haskell.  We follow that by a section
discussing three famous logic programming problems: \texttt{append}, type
inference, and a generalizaton of Prolog's \texttt{name}.  We conclude with
some perspective on why this approach works.

\section{Preliminaries}

There are three categories of operations we need.  The first category
contains operators for creating lexically-scoped logic variables
(\texttt{exists}), currying (\texttt{lambda@} and \texttt{@}), and a
lexical binder for multiple values (\texttt{let-values}).  The second
category concerns itself with building (\texttt{empty-subst},
\texttt{compose-subst}, and \texttt{extend-subst}), and applying
(\texttt{subst-in}) substitutions.  The final category has 
operations for unifying two terms: \texttt{unify} and \texttt{unify*}.

\subsection{Creating variables, Currying, and a macro for multiple values}

A data structure that may be a \emph{logic} variable or contain such
variables is called a \emph{term}.  We create a logic variable using
the procedure \texttt{var}, which takes a symbol as its argument. Here
is an example of its use.

\begin{alltt}
> (let ([x (var 'x)] [xx (var 'x)])
    (list (var-name x) (var-name xx) (var? x) (eqv? x xx)))

(x x #t #f)
\end{alltt}

\noindent
There is a procedure \texttt{var-name} that retrieves the name that a
variable has been built from, there is a predicate \texttt{var?} that
distinguishes variables from other data structures, and each
invocation of \texttt{var} makes a new one.  (We leave the details of
defining these three procedures as an exercise.  Certainly, a uniquely
tagged \texttt{cons} pair with its name in the \texttt{cdr} could be
used, but most Scheme systems have their own record facility.  For
purists, instead of relying on the ability to distinguish two
\texttt{cons} cells with \texttt{eqv?}, one could associate a unique
timestamp with each logic variable.)

We introduce a macro, \texttt{exists}, which allows us to both create
logic variables and have them known within a lexical scope.  Since
we are generating a \texttt{let} expression, it is necessary to
include different names in the first argument to \texttt{exists}.

\begin{alltt}
(define-syntax exists
  (syntax-rules ()
    [(_ (name ...) body0 body1 ...)
     (let ([name (var 'name)] ...) body0 body1 ...)]))
\end{alltt}

\noindent
For example,
\begin{alltt}
(exists (x y z)
  (list x 1 y 2 z 3))
  
\end{alltt}

\noindent
expands to (is the same as writing)

\begin{alltt}
(let ([x (var 'x)] [y (var 'y)] [z (var 'z)])
  (list x 1 y 2 z 3))
\end{alltt}

Here is how we can verify these ideas using \texttt{expand} and
\texttt{expand-only}.

\begin{alltt}
> (expand 
    '(exists (x y z)
       (list x 1 y 2 z 3)))

((lambda (x y z) (list x 1 y 2 z 3))
 (var 'x)
 (var 'y)
 (var 'z))

> (expand-only '(exists)
    '(exists (x y z)
       (list x 1 y 2 z 3)))

(let ([x (var 'x)] [y (var 'y)] [z (var 'z)])
  (list x 1 y 2 z 3))
\end{alltt}

There are other uses of \texttt{expand-only}, below.  The main thing
to observe, however, is that we do not have to expand
\texttt{let}-expressions, if we don't want to.  The deeper the nesting
levels of macro calls, the fewer items we might want to place in the
first argument to \texttt{expand-only}.  This one only has two levels,
\texttt{exists} expands to \texttt{let} and \texttt{let} expands to an
application of a \texttt{lambda}-expression.

One other item that we should get familiar with is that using \texttt{list}
works, but misses the structural relationship that we get back using
\texttt{quasiquote} and \texttt{unquote}.  The last example can also be
written like this,

\begin{alltt}
> (expand-only '(exists)
    '(exists (x y z)
       `(,x 1 ,y 2 ,z 3)))

(let ([x (var 'x)] [y (var 'y)] [z (var 'z)])
  `(,x 1 ,y 2 ,z 3))
\end{alltt}

This way we get Scheme's quasiquote ``\texttt{`}'' and unquote
``\texttt{,}'' to indicate where the logic variables can be found in a
list structure.  But, as we discover there is still much we can do
with logic programming that does not involve lists.  In fact, they
don't appear until the last section.

\subsubsection{Currying macros}

We also define a curried \texttt{lambda}, \texttt{lambda@},

\begin{alltt}
(define-syntax lambda@
  (syntax-rules ()
    [(_ () body0 body1 ...) (begin body0 body1 ...)]
    [(_ (formal0 formal1 ...) body0 body1 ...)
     (lambda (formal0)
       (lambda@ (formal1 ...) body0 body1 ...))]))
\end{alltt}

To go along with \texttt{lambda@}, we have \texttt{@}, a curried
application macro.

\begin{alltt}
(define-syntax @    
  (syntax-rules ()
    [(_ arg0) arg0]
    [(_ arg0 arg1 arg2 ...) (@ (arg0 arg1) arg2 ...)]))
\end{alltt}

\begin{alltt}
> (@ (lambda@ (x y z) (+ x (+ y z))) 1 2 3)

6

> (expand '(@ (lambda@ (x y z) (+ x (+ y z))) 1 2 3))

((((lambda (x)
     (lambda (y)
       (lambda (z)
         (+ x (+ y z)))))
   1) 2) 3)
\end{alltt}

Curried application \texttt{@} and curried \texttt{lambda@} give us
the ability to build and invoke our functions so that the arguments
arrive one at a time.

\subsubsection{A macro for binding multiple return values}

There are times when it is better to be explicit about returning
multiples values.  Here is a simple macro that should encourage
such behavior.

\begin{alltt}
(define-syntax let-values
  (syntax-rules ()
    [(_ (x ...) vs body0 body1 ...)
     (call-with-values 
       (lambda () vs)
       (lambda (x ...) body0 body1 ...))]))

> (let-values (x y z) (values 1 2 3) 
    (+ x (* y z)))

7
\end{alltt}

Since \texttt{values} is a global function it can be called from
anywhere,
\begin{alltt}
> (= (let ([f (lambda () values)])
       (let-values (x y z) ((f) 1 2 3)
         (+ x (* y z))))
     (let ([f (lambda (w) (values w (+ w 1) (+ w 2)))])
       (let-values (x y z) (f 1)
         (+ x (* y z)))))

#t
\end{alltt}

\subsection{Substitutions}

A substitution, \texttt{s}, is a list of real commitments, represented
using an association list. A \emph{commitment} is a pairing
(\texttt{commitment}) of a variable (\texttt{commitment->var}) to a
term (\texttt{commitment->term}) and it is \emph{real} if its variable
is not the same as its term.  We will have occasion to naively add a
commitment to a substitution, so we must be certain that the
commitment is real.

\begin{alltt}
(define cons-if-real-commitment
  (lambda (var term subst)
    (cond
      [(eqv? term var) subst]
      [else (cons (commitment var term) subst)])))
\end{alltt}

\noindent
The variables (i.e., \texttt{(map commitment->var s)}) of a
substitution must form a set.  We introduce two kinds of
substitutions.  The first is the empty substitution:
\texttt{empty-subst}, which we represent with the empty list.  The
second, built with \texttt{compose-subst}, is formed by refining one
substitution with another to form a new substitution.
\footnote{This material on substitutions is derived from
definitions and examples on pages 18 and 19 of J.\ W.\ Lloyd's
\emph{Foundations of Logic Programming}.}

\begin{alltt}
(define empty-subst '())
\end{alltt}

\begin{alltt}
(define compose-subst
  (lambda (base refining)
    (let refine ([base base] [survivors refining])
      (cond
        [(null? base) survivors]
        [else (cons-if-real-commitment
                (commitment->var (car base))
                (subst-in (commitment->term (car base)) refining)
                (refine (cdr base)
                  (cond
                    [(assv (commitment->var (car base)) survivors)
                     => (lambda (c) (remv c survivors))]
                    [else survivors])))]))))
\end{alltt}
When does \texttt{compose-subst} behave like \texttt{append}?  There
are two aspects of the code to consider.  First, the \texttt{assv}
test must always fail.  For if it doesn't, \texttt{survivors}, which
starts out as \texttt{refining} shrinks. Second, the \texttt{subst-in}
expression must not yield a new term the same as
\texttt{(commitment->var (car base))}.  One way to do this would be to
make sure that the variables of \texttt{(commitment->term (car base))}
do not overlap with \texttt{(map commitment->var refining)}.
\newpage
But, of course, \texttt{compose-subst} rarely behaves exactly like
\texttt{append}.  A commitment in the original \texttt{refining}
substitution may not show up in the resultant substitution.  This
happens when a variable bound in \texttt{base} is also bound in
\texttt{refining}.  This should be obvious, since no variable is bound
more than once in the resultant substitution.  The one bound in
\texttt{base} is the one that matters.  So composing two substitutions
is like taking the union of the commitments relative to the
variables and utilizing the content of the refining substitution to
affect the term in \texttt{base}'s substitution.  Any resultant
unreal commitments are excluded from the resultant substitution.

Suppose we have a variable and a term that we would like to naively
include in a substitution.  If the variable is unbound, the
variables in the term are not in \texttt{(map commitment->var subst)},
and if the term is not the same as the variable, then we can use this
simple definition.

\begin{alltt}
(define extend-subst
  (lambda (var t refining)
    (cons (commitment var t) refining)))
\end{alltt}

This special case arises often enough that we feel obligated to use it
later.  But, we must convince ourselves that it is the right
definition.  Clearly, we must be certain that its first two arguments
are not the same, for if they were, the result would be the original
substitution.  Also, if the first argument is bound, then we know that
the associated term would be removed from the original substitution.
Finally, if the second argument contains any bound variables, their
associated terms would affect the term that would be bound in the
extended substitution.  Obviously, all constants qualify, but later we
see that when we include pairs, some of them may qualify, too.

The procedure \texttt{subst-in}, below, translates everything to
itself except variables.  If the variable is in the variables of the
substitution, its associated term is returned, otherwise the variable
itself is returned, since it is virtually associated with
itself. Initially, we restrict terms to be variables or values that
can be compared with \texttt{eqv?}.  In the last section we
\emph{enlarge the set of terms to include pairs (lists), which may
contain variables.}

\begin{alltt}
(define subst-in
  (lambda (t subst)
    (cond
      [(var? t)
       (cond
         [(assv t subst) => commitment->term]
         [else t])]
      [else t])))
\end{alltt}

\texttt{Exercise 1:} Rewrite \texttt{compose-subst} and
\texttt{subst-in} to check in advance to see if any of
its substitution arguments are empty.

In the definition of \texttt{unify*}, below, we create a
substitution of exactly one real commitment. We make this explicit
with the definition of \texttt{unit-subst}, below. The outer
\texttt{list} in the definition is there because every substitution is a
\emph{list} of commitments.  

\begin{alltt}
(define unit-subst 
  (lambda (var t)
    (list (commitment var t))))
\end{alltt}

Consider the use of substitutions in everyday experiences.  Before you
buy your first motorized vehicle, you have made no commitments to
yourself about its purchase.  You have the empty substitution.  Then
you decide to buy a four-wheeled motorized vehicle.  Now you have
enlarged your empty substitution to include a single commitment that
whatever you buy should have four wheels and an engine.  Then, you
decide to purchase a car.  You have refined your earlier commitment to
buy a truck, a car, or some other four-wheeled vehicle to buying a
car.  You still have only one commitment in your substitution, though
you have composed two nonempty substitutions: the one that stated that
you would buy a four-wheeled motorized vehicle and the one that stated
that you would buy a car.  You can refine your current substitution by
stating that the car would not be over three years old.  You still
have only one commitment.  At some point, you can \emph{also} commit
to purchasing a television.  Now, you have two commitments.  Each time
you refine these two commitments you get a substitution with two
commitments.  For example, you might choose to get a sedan and a color
television.  You still don't know what you are going to get, but as
your current substitution is refined, you are getting closer and
closer to making your decision.  Let's consider some examples:

\begin{alltt}
> (exists (x y)
    (equal?
      (compose-subst (unit-subst x y) (unit-subst y 52))
      `(,(commitment x 52) ,(commitment y 52))))
#t

> (exists (w x y)
    (equal?
      (let ([s (compose-subst (unit-subst y w) (unit-subst w 52))])
        (compose-subst (unit-subst x y) s))
      `(,(commitment x 52) ,(commitment y 52) ,(commitment w 52))))
#t

> (exists (w x y)
    (equal?
      (let ([s (compose-subst (unit-subst w 52) (unit-subst y w))])
        (compose-subst (unit-subst x y) s))
    `(,(commitment x w) ,(commitment w 52) ,(commitment y w))))
#t
\end{alltt}
\newpage
\begin{alltt}
> (exists (w x y)
    (equal?
      (let ([s (compose-subst (unit-subst y w) (unit-subst x y))]
            [r (compose-subst
                 (compose-subst (unit-subst x 'a) (unit-subst y 'b))
                 (unit-subst w y))])
        (compose-subst s r))
      `(,(commitment x 'b) ,(commitment w y))))
#t
\end{alltt}
In the first example, the base substitution commits \texttt{x} to
\texttt{y}.  Then, the refining substitution commits \texttt{y} to
\texttt{52}, refining \texttt{x} to \texttt{52}.  In the second
example, the base substitution of \texttt{s} commits \texttt{y} to
\texttt{w}, refining \texttt{y} to \texttt{52}.  Then \texttt{s} is
used as the refining substitution, so the resultant substitution
commits \texttt{x} to \texttt{52}.  In the third example, the refining
substitution of \texttt{s} does not influence its base.  Thus
\texttt{s} contains the two commitments.  Then the base substitution
of the resultant substitution is refined by the \texttt{y}-commitment committing
\texttt{x} to \texttt{w}.  In the fourth example, we construct two
substitutions manually.  The first contains two commitments and the
second contains three commitments.  The second refines the first to
yield a substitution with only two commitments, since we attempt to
create a commitment of \texttt{y} to \texttt{y}.

\texttt{Exercise 2:} Hand trace the fourth example, above, and
thus prove that composing a substitution with two commitments and
a substitution with three commitments can yield a substitution
with just two commitments.

\subsection{Unification}

We introduce the unification interface: \texttt{unify},
\texttt{unify*}.  The interface \texttt{unify}, below, tries to find a
substitution that will treat the two substituted for terms as equal if
the resultant substitution were applied to them.  If successful, it
returns a composed substitution.  If it cannot unify the two terms,
then false is returned.  Since we have a very limited definition of
term at this time, we can easily write the auxiliary procedure
\texttt{unify*}, below.  Two terms unify if they are the same
variable, the same constant term, or one of them is the
\emph{anonymous} variable.  Then, \texttt{unify*} returns the empty
substitution.  That is why we do not choose false to represent the
empty substitution.  Returning the empty substitution means that the
two terms unified.  Of course, that would mean that the two terms
contained no variables or virtually contained no variables.
Otherwise, if either term is a variable, it treats the other as a term
and returns a substitution of a singleton commitment. In all other
cases, the unifier returns false.  (We again stress that these
definitions work until we get to the last section, where we add pairs,
which can contain variables.  When we make this change, we also must
change \texttt{subst-in} to support pairs.
\newpage
\begin{alltt}
(define _ (exists (_) _))

(define unify
  (lambda (t u subst)
    (cond
      [(unify* (subst-in t subst) (subst-in u subst))
       => (lambda (refining-subst)
            (compose-subst subst refining-subst))]
      [else #f])))
\end{alltt}

\begin{alltt}
(define unify*
  (lambda (t u)
    (cond
      [(trivially-equal? t u) empty-subst]
      [(var? t) (unit-subst t u)]
      [(var? u) (unit-subst u t)]
      [else #f])))
\end{alltt}

\begin{alltt}
(define trivially-equal?
  (lambda (t u)
    (or (eqv? t u)
        (and (string? t) (string? u) (string=? t u))
        (eqv? t _)
        (eqv? u _))))
\end{alltt}

Here are a few examples.

\begin{alltt}
> (exists (x y)
    (and
      (equal? (unify x 3 empty-subst) `(,(commitment x 3)))
      (equal? (unify 4 y empty-subst) `(,(commitment y 4)))
      (equal? (unify x y empty-subst) `(,(commitment x y)))
      (equal? (unify 'x 'x empty-subst) empty-subst)
      (equal? (unify x x empty-subst) empty-subst)
      (not (unify 4 'y empty-subst))
      (not (unify 'x 3 empty-subst))
      (not (unify 3 4 empty-subst))))

#t
\end{alltt}

Only the fifth example is interesting.  It returns the empty
substitution, since the two terms are the same variable.  The sixth
and seventh don't unify because a symbol (not a variable) can never be
equal to a number.

\section{The unsullied logic system}

The logic system has just two primary operators: \texttt{relation}
(and \texttt{fact}, which is derived from \texttt{relation}) which
expands into a \texttt{lambda} expression, and
\texttt{extend-relation}, which gives us the ability to form a new
relation from other relations.  In addition, we have two macros for
handling sequences of antecedents: \texttt{all} (and \texttt{all!},
which is derived from \texttt{all}.)  and its dual \texttt{any}, which
is derived from \texttt{relation} and \texttt{extend-relation}.  To
enter the system, we have a procedure \texttt{query}, and two
interfacing operations: \texttt{solve} and \texttt{solution}, which is
derived from \texttt{solve}.

Here are the types we use in this system:

\begin{alltt}
                  Fk = () -> Ans
                 Ans = Nil + [Subst,Fk]
                Cutk = Fk
                  Sk = Fk -> Subst -> Cutk -> Ans
          Antecedent = Sk -> Sk
            Relation = Term* -> Antecedent
\end{alltt}

\texttt{Ans} is a stream of substitutions, since \texttt{Fk} is a
function of zero arguments. An \texttt{Antecedent} is an \texttt{Sk}
transformer. 

\subsection{It's a small world}

Let's define a single fact. (Warning: This is not the best time to try
to figure out these definitions, but if you want to, the whole point
of passing \texttt{()} from top rule of \texttt{relation} to the lower
rules is to act as an accumulator.  We are trying to get a list of
variables the same length as the number of arguments following the
keyword \texttt{to-show}. Once we have that, which happens in the
second rule of \texttt{relation} when one of the arguments is
\texttt{()}, then we generate the big \texttt{lambda} expression.  The
body of the \texttt{exists} expression is an \texttt{all} expression.
Its first operand is an \texttt{all!} expression that installs
\texttt{tight-cut} before and after each \texttt{(== g x)} antecedent.
Because \texttt{tight-cut} always succeeds, when going left to right,
it can be ignored.  But, if one of the \texttt{(== g x)} fails to
unify, then the system backs into the \texttt{fk}, which is the
argument fed to \texttt{!!}, whose last two arguments have been $\eta$
reduced. This guarantees that while it is unifying several terms, one
mismatch will abort the outer \texttt{all} expression, If it succeeds,
it processes the remaining antecedents under the substitution returned
from the completed unification.  We ignore \texttt{short-cut} for
now. Come back and reread this after you see how \texttt{relation}
expands.  It will seem much simpler, then.)

\begin{alltt}
(define-syntax all
  (syntax-rules ()
    [(_) (lambda (sk) sk)]
    [(_ ant) ant]
    [(_ ant0 ant1 ...)
     (lambda (sk)
       (ant0 ((all ant1 ...) sk)))]))
\end{alltt}

The \texttt{all} expression has two things in common with Scheme's
\texttt{and} expression.  The second rule is superfluous in both.
Also, the variant with no arguments is a constant.  With \texttt{and}
it is true and with \texttt{all} it is success.  Later we discuss the
difference between succeeding and being true.

\begin{alltt}
(define-syntax all!
  (syntax-rules ()
    [(_ (cutter ant ...)) (all cutter ant ...)]
    [(_ (cutter ant ...) ant0 ant1 ...)
     (all! (cutter ant ... ant0 cutter) ant1 ...)]))
\end{alltt}

The macro \texttt{all!} generates an \texttt{all} expression with the
\texttt{ant}\emph{i} separating the \texttt{cutter}s.  We generally
expect that \texttt{cutter} is a lexically-bound variable whose value comes
from \texttt{short-cut} or \texttt{tight-cut}, below.  Here is a meaningless
example that clarifies the simplicity of \texttt{all!}.

\begin{alltt}
> (expand-only '(all!) 
    '(all! (tight-cut) 1 2 3 4 5))

(all tight-cut 1 tight-cut 2 tight-cut 3 tight-cut 4 tight-cut 5)
\end{alltt}

\begin{alltt}
(define-syntax relation
  (syntax-rules (to-show)
    [(_ short-cut (ex-id ...) (to-show x ...) ant ...)
     (relation short-cut (ex-id ...) () (x ...) (x ...) ant ...)]
    [(_ short-cut (ex-id ...) (g ...) () (x ...) ant ...)
     (lambda (g ...)
       (lambda@ (sk fk subst cutk)
         (@ (let ([tight-cut (!! fk)] [short-cut (!! cutk)])
              (exists (ex-id ...)
                (all (all! (tight-cut) (== g x) ...) ant ...)))
            sk fk subst cutk)))]
    [(_ short-cut ex-ids (var ...) (x0 x1 ...) xs ant ...)
     (relation short-cut ex-ids (var ... g) (x1 ...) xs ant ...)]))
\end{alltt}
\begin{alltt}
(define !!
  (lambda (exiting-fk)
    (lambda@ (sk fk)
      (@ sk exiting-fk))))
\end{alltt}
\begin{alltt}
(define ==
  (lambda (t u)
    (lambda@ (sk fk subst cutk)
      (cond
        [(unify t u subst)
         => (lambda (subst)
              (@ sk fk subst cutk))]
        [else (fk)])))))
\end{alltt}
\newpage
We now have enough tools to define a relation with a single fact.
Our relation says that ``Pete is the father of Sal.''

\begin{alltt}
(define father  
  (relation _ ()
    (to-show 'pete 'sal)))
\end{alltt}

Before continuing, let's observe that we almost have an
$\eta$-reduction opportunity.  If the variables, \texttt{fk} and
\texttt{cutk} were not used, then the fact that the relation absorbs
four arguments and then passes them along would qualify as an $\eta$
redex.  (\texttt{(lambda (x) (}\emph{M}\texttt{ x))} $\eta$-reduces to
\emph{M} provided that \texttt{x} is not free in \emph{M} and
\texttt{M} obviously terminates.  Variables and functions obviously
terminate, but so do lots of other expressions.)  So, we use a trick
to hide the lost opportunity.  We pass a \texttt{lambda} expression
that binds these two needed variables, \texttt{tight-cut} and
\texttt{short-cut}, to another function \texttt{reify-cutters}, below,
and we let it accept the lost opportunity.  This uncluttering of
\texttt{relation} aids our ability to read expanded code, but
obviously it is not necessary for any other reason.

\begin{alltt}
(define-syntax relation
  (syntax-rules (to-show)
    [(_ short-cut (ex-id ...) (to-show x ...) ant ...)
     (relation short-cut (ex-id ...) () (x ...) (x ...) ant ...)]
    [(_ short-cut (ex-id ...) (g ...) () (x ...) ant ...)
     (lambda (g ...)
       (reify-cutters
         (lambda (tight-cut short-cut)
           (exists (ex-id ...)
             (all (all! (tight-cut) (== g x) ...) ant ...)))))]
    [(_ short-cut ex-ids (var ...) (x0 x1 ...) xs ant ...)
     (relation short-cut ex-ids (var ... g) (x1 ...) xs ant ...)]))

(define reify-cutters
  (lambda (proc)
    (lambda@ (sk fk subst cutk)
      (@ (proc (!! fk) (!! cutk)) sk fk subst cutk))))
\end{alltt}

Let's focus on the single fact that says that ``Pete is the father of
Sal.''  But, what it really says is that there is a relation, that
joins ``Pete to Sal,'' which we are calling ``The father relation.''
And when do we know that Pete is the father of Sal?  When we have
shown that all the expressions below the \texttt{to-show} keyword
expresion hold.  But, since there are none, it holds vacuously.
Therefore, Pete is always the father of Sal.  But, if we have another
relation, \emph{older than}, we can have the fact that ``Pete is older
than Sal.''  The only thing that would change is that we would replace
the variable \texttt{father} by \texttt{older-than}.  This is a lot
deeper than it may first appear.  Our relations are first class.  If
we wanted to do the same thing, we could also just \texttt{(define
older-than father)}, which would say that ``Pete is older than Sal.''
Of course, that should not come as a surprise, since Pete is Sal's
father, but these two relations are completely disjoint!  For example,
we may be referring to four different people: two Petes and two Sals.

Sometimes the list of variable names that follows \texttt{relation} is
nonempty, but don't be confused: \emph{This list does not tell how
many arguments the relation takes. That is determined by the number of
terms following the keyword} \texttt{to-show}.

Consider the partial expansion of \texttt{father}
\begin{alltt}
> (expand-only '(relation)
    '(define father  
       (relation _ ()
         (to-show 'pete 'sal))))
\end{alltt}

\begin{alltt}
(define father
  (lambda (g1 g2)
    (reify-cutters
      (lambda (tight-cut _)
        (exists ()
          (all (all! (tight-cut) (== g1 'pete) (== g2 'sal))))))))
\end{alltt}

The relation \texttt{father} is represented as a procedure of two
arguments, \texttt{g1} and \texttt{g2}.\footnote{The actual expansion
produces \texttt{g} and \texttt{g}.  But, they are different
variables.} It can figure that out from the number of operands in
the \texttt{to-show} keyword expression. The \texttt{reify-cutters}
clause binds two variables.  The first variable to be bound is
\texttt{tight-cut}.  Its value is an antecedent that always succeeds in the
forward direction, but invokes a failure continuation as it backtracks
into \texttt{tight-cut}.  The second one, can be any variable we pass as
the first operand to \texttt{relation}, and since we are not concerned
about \emph{cut} yet, we pass the innocuous underscore variable.
Later we pass \texttt{cut}.  Those who are enamored of Prolog's
notation might pass \texttt{!}.  Also, the ``\texttt{(exists ()}'' is
useless.  If we can unify the value of the variable \texttt{g1} with
the value of \texttt{'pete} and the value of the variable \texttt{g2}
with the value of \texttt{'sal}, then the resultant substitution is
passed to a procedure that addresses any remaining antecedents.  In
our simple example, there are none.

To make things a little clearer, we expand the expression with respect
to \texttt{all!}.

\begin{alltt}
(define father
  (lambda (g1 g2)
    (reify-cutters
      (lambda (tight-cut _)
        (exists ()
          (all (all tight-cut (== g1 'pete) tight-cut (== g2 'sal) tight-cut)))))))
\end{alltt}

\texttt{Exercise 3:} Prove that the outer \texttt{all} is not
necessary in the latest definition of \texttt{father}.

Whenever \texttt{father} is invoked, a dad (\texttt{g1}) and a child
(\texttt{g2}) get bound to a goal term, which is then transformed into
a goal function.  A failure continuation is invoked if the arguments
fail to unify.  This effects a return from the call to
\texttt{father}.  Therefore, when it fails to find a match, it returns
from the call.

Next, we define a relation \texttt{child-of-male} and compare its expansion
to the previous relation.

\begin{alltt}
(define child-of-male
  (relation _ (child dad)
    (to-show child dad)
    (father dad child)))
\end{alltt}

\begin{alltt}
> (expand-only '(relation)
    '(define child-of-male
       (relation _ (child dad)
         (to-show child dad)
         (father dad child))))

(define child-of-male
  (lambda (g1 g2)
    (reify-cutters
      (lambda (tight-cut _)
        (exists (child dad)
          (all (all! (tight-cut) (== g1 child) (== g2 dad))
            (father dad child)))))))
\end{alltt}

%Such a relation is often written using a horizontal line,
%\Rule{}{\texttt{(father dad child)}} {\texttt{(child-of-male child dad)}}{}

We can see that to show that some person (\texttt{child}) is the child
of a male (\texttt{dad}), we have to show that \texttt{dad} is the
father of \texttt{child}.  This is very naive but logical
reasoning.\footnote{Use the partially-expanded variant of
\texttt{child-of-male} instead of the unexpanded \texttt{relation} to
install a \texttt{let} exprssion around the \texttt{reify-cutters}
expression.  For example, one might want to read from a file, print
what was read, and then unify against it instead of against
\texttt{child}.}

\subsection{Testing \texttt{father} and \texttt{child-of-male}}

We are now ready to test \texttt{father}.  For example, we might wish
to determine ``If Pete is the father of Sal.''

\begin{alltt}
> (let ([initial-sk (lambda@ (fk subst cutk) (cons subst fk))]
        [initial-fk (lambda () '())]
        [empty-subst '()])
    (@ (father 'pete 'sal) initial-sk initial-fk empty-subst initial-fk))
(() . #<procedure initial-fk>)
\end{alltt}

If the goal succeeds by invoking \texttt{initial-sk} on a substitution
and two failure continuations, then once it gets the second failure
continuation, it returns a pair of the substitution and the
first continuation. If the goal fails, the empty list, the result of
invoking \texttt{initial-fk}, is returned.

Since our test has no variables, we know that we do not refine the
original substitution, but unify when the raw values in each term are
the same.  What is the purpose of the empty substitution, \texttt{()}?
At this point, we could apply (using \texttt{subst-in} the empty
substitution to each term in the original term and produce what we
started with: \texttt{(pete sal)}, but with the assurance that
\texttt{pete} has been shown to be the father of \texttt{sal}.

All the arguments (i.e., \texttt{initial-sk}, \texttt{initial-fk}, and
\texttt{empty-subst}) to \texttt{(father 'pete 'sal)} are
constants. Therefore, we abstract this with the function
\texttt{query}, which packages these initial values with the
antecedent, the result of the \texttt{father} call.

\begin{alltt}
(define query
  (let ([initial-fk (lambda () '())]
        [initial-sk (lambda@ (fk subst cutk) (cons subst fk))]
        [empty-subst '()])
    (lambda (ant)
      (@ ant initial-sk initial-fk empty-subst initial-fk))))
\end{alltt}

We next consider the role of nonempty substitutions.  Instead of
asking a specific question about Pete's relationship to Sal, let's
determine a child of Pete.  To do this, we introduce a logic variable.

\begin{alltt}
> (exists (x)
    (let ([result (query (father 'pete x))])
      (and
        (equal? (car result) `(,(commitment x 'sal)))
        (null? ((cdr result))))))
#t
\end{alltt}

\begin{alltt}
> (exists (x)
    (let ([result (query (father 'pete x))])
      (let ([subst (car result)])
        (list (subst-in 'pete subst) (subst-in x subst)))))
(pete sal)
\end{alltt}

\noindent
We see that the resultant substitution binds \texttt{x} to \texttt{sal}.
When we use this substitution, \texttt{pete} becomes \texttt{pete},
but \texttt{x} becomes \texttt{sal}.

We can test \texttt{child-of-male} the same way, but this time we
choose to give it no raw data, just variables, and hope that a
nonempty substitution, the \texttt{car} of the query, is returned.

\begin{alltt}
> (exists (x y)
    (let ([result (query (child-of-male x y))])
      (let ([subst (car result)])
         (list (subst-in x subst) (subst-in y subst)))))

(sal pete)
\end{alltt}

And, we discover that Sal is the child of Pete.

\subsection{Generating more than one answer}

The difference between a relation and a function is that with a
function, only one answer is associated with an input, but with a
relation the same input can lead to many answers.  Now, we demonstrate
in what ways functions such as the definition of \texttt{father},
below, can be treated as relations.

Suppose that Pete is also the father of Pat, what changes could be
made to get both answers?  First, we introduce
\texttt{binary-extend-relation} to allow for an additional fact.

\begin{alltt}
(define-syntax binary-extend-relation
  (syntax-rules ()
    [(_ (id ...) rel-exp1 rel-exp2)
     (let ([rel1 rel-exp1] [rel2 rel-exp2])
       (lambda (id ...)
         (lambda@ (sk fk subst cutk)
           (@ (rel1 id ...)
             sk (lambda () (@ (rel2 id ...) sk fk subst cutk)) subst cutk))))]))
\end{alltt}

\begin{alltt}
(define pete-pat
  (relation _ ()
    (to-show 'pete 'pat)))
\end{alltt}

Now, we have two relations.  What we want to do is combine the two
relations into a single one.  

\begin{alltt}
(define father (binary-extend-relation (a1 a2) father pete-pat))
\end{alltt}

The operator \texttt{binary-extend-relation} takes two relations and a
list of \texttt{lambda} variables, the length of which is the arity of
the relation.  Thus, if you follow the convention we use, the \emph{i}
in the last \texttt{a}\emph{i} in the list is its arity.  It tries to
find a result in \texttt{rel1}.  If it succeeds, it returns a
substitution along with a failure continuation, which is to try to
find a result in \texttt{rel2}.  If it fails, then it invokes the just
described failure continuation.  The early evaluation of the relation
expressions are critical in order to guarantee that we can write
expressions such as \texttt{(define x (extend-relation (a1 a2) x y))}.
Without this early evaluation, the original relation \texttt{x} would
not be part of the new definition of \texttt{x}, which would lead to
an infinite loop.  This is an argument for making
\texttt{binary-extend-relation} a procedure, but then it would be
necessary to use \texttt{(lambda args ...)} and \texttt{apply}, which
we belieive should be avoided, when possible.

A convenient abbreviation of a relation where there is no antecedent
is decribed by the macro \texttt{fact}.

\begin{alltt}
(define-syntax fact
  (syntax-rules ()
    [(_ (ex-id ...) x ...) (relation _ (ex-id ...) (to-show x ...))]))
\end{alltt}

Then we could write the \texttt{father} relation like this.

\begin{alltt}
(define father
  (binary-extend-relation (a1 a2)
    (fact () 'pete 'sal)
    (fact () 'pete 'pat)))
\end{alltt}

\begin{alltt}
> (exists (x)
    (let ([result (query (father 'pete x))])
      (let ([subst (car result)]
            [fk (cdr result)])
      (cons
        (list (subst-in 'pete subst) (subst-in x subst))
        (let ([result (fk)])
          (let ([subst (car result)]
                [fk (cdr result)])
            (cons
              (list (subst-in 'pete subst) (subst-in x subst))
              (let ([result (fk)])
                (if (null? result)
                  '()
                  (error 'father "Pete only has two children"))))))))))

((pete sal) (pete pat))
\end{alltt}
\noindent
First, we add that Pete is also the father of Pat.  Then, when we
invoke \texttt{fk} the first time, we get another result.  When we
invoke it a second time, we are out of results, so we get back the
empty list, the result of invoking the initial failure continuation.

\subsection{Streams (infinite lists) provide a natural interface}

We would like to abstract the previous program in a more coherent way.
Later, we see an example where there is no limit on the number of
answers, but if we want to process the answers as a list, we must
place some bound on the size of the list.

The \texttt{(car result)} is a substitution, and the \texttt{(cdr
result)} is a thunk that if invoked returns a value whose \texttt{car}
is a substitution, and whose \texttt{cdr} is a thunk, etc.  This is a
stream (possibly infinite list) of substitutions.  We write the
function, \texttt{stream-prefix}, which given a bound \texttt{n}, and
a stream, \texttt{strm}, yields a list with at most one element, if
\texttt{n} is 0; at most two elements, if \texttt{n} is 1; etc.  But,
we must be careful that going for the \texttt{n+1}st element of a
stream does not lead to an infinite loop, even though we only want the
first \texttt{n} elements! This causes \texttt{stream-prefix} to be
written in a slightly awkward fashion.

\begin{alltt}
(define stream-prefix
  (lambda (n strm)
    (if (null? strm) '()
      (cons (car strm)
        (if (zero? n) '()
          (stream-prefix (- n 1) ((cdr strm))))))))
\end{alltt}

Before we go any further, we have some details about displaying
answers that needs addressing.  This will seem a bit bizarre at first
reading, but there is only one reason for all this code: \emph{Logic
variables are not eye-pleasing}.  So, we want to devise a good way to
display them.  Normally, a logic system uses a \texttt{gensym} name,
but then the results are very difficult to follow.  So rather than use
generated names, we construct our own artificial names.

\begin{alltt}
(define artificial-name
  (lambda (var-c)
    (string->symbol
      (string-append
        (symbol->string (var-name (car var-c))) "." (number->string (cdr var-c))))))

> (artificial-name `(,(exists (who) who) . 5))

who.5
\end{alltt}
We associate a unique artificial name with each logic variable.  Each
logic variable, however, only stores a symbol within it.  Two
different logic variables could easily have the same symbol within it,
so there needs to be a mechanism to have two different artificial
names.  The way we solve this problem is to use an environment that is
both an input and an output result.  Each item in the environment
binds a logic variable to a counter.  If the logic variable is found
in the environment, then the associated artificial name is returned.
If it is not found in the environment, then a second search using
\texttt{assv/var-name}, below, is attempted.  If a logic variable built
from the same name is in the environment, then a new item is added to
the environment, but its counter is one larger than the previous one
If these two attempts fail, then the new item added to the envionment
starts the counter at zero.  This is the job of \texttt{concretize-var}, below.
\begin{alltt}
(define concretize-var
  (lambda (var env)
    (cond
      [(assv var env)
       => (lambda (var-c) (values (artificial-name var-c) env))]
      [else (let ([var-c `(,var . ,(cond
                                     [(assv/var-name (var-name var) env)
                                      => (lambda (var-c) (+ (cdr var-c) 1))]
                                     [else 0]))])
              (values (artificial-name var-c) (cons var-c env)))])))
\end{alltt}
\begin{alltt}
(define assv/var-name
  (lambda (name env)
    (cond
      [(null? env) #f]
      [(eqv? (var-name (caar env)) name) (car env)]
      [else (assv/var-name name (cdr env))])))
\end{alltt}

\noindent
The only thing to notice about this code is that two items are returned
using \texttt{values} as the result of any call to
\texttt{concretize-var}.

Because we are in a logic system, we don't know if we are concretizing
a variable or a trivial constant.  Thus, we use a driver
\texttt{concretize-term}, which dispatches appropriately.  When we add
pairs as a kind of term, then we need to extend this definition.

\begin{alltt}
(define concretize-term
  (lambda (t env)
    (cond
      [(var? t) (concretize-var t env)]
      [else (values t env)])))
\end{alltt}

In keeping with our philosophy of avoiding gratuitous lists, we have
the macro \texttt{concretize-sequence}, which builds a list of
concretized terms from a bunch of terms.  

\begin{alltt}
(define-syntax concretize-sequence
  (syntax-rules ()
    [(_ t0 ...) (concretize-sequence-aux '() t0 ...)]))

(define-syntax concretize-sequence-aux
  (syntax-rules ()
    [(_ env) '()]
    [(_ env t0 t1 ...)
     (let-values (ct new-env) (concretize-term t0 env)
       (cons ct (concretize-sequence-aux new-env t1 ...)))]))
\end{alltt}

Let us return to our discussion about how we are going to use
\texttt{stream-prefix}.  Once we have a finite list of substitutions,
we are free to use them however we wish. For example, we define a
macro \texttt{solve} that takes a positive integer upper bound and a
relation call.  It returns a list like the one above for
\texttt{pete}'s children.  Each variable is replaced by the
associated term of its commitment.

\begin{alltt}
(define-syntax solve
  (syntax-rules ()
    [(_ n (rel t ...))
     (map (lambda (subst)
            (concretize-sequence (subst-in t subst) ...))
       (stream-prefix (- n 1) (query (rel t0 ...))))]))
\end{alltt}

\begin{alltt}
> (exists (x) (solve 5 (father 'pete x)))

((pete sal) (pete pat))
\end{alltt}

\noindent
Of course, the resultant list contains only two answers, but
asking for five does no harm, since it quits early when the
\texttt{null?} test in \texttt{stream-prefix} holds.  
\newpage
Let's see what this call partially expands to.
\begin{alltt}
> (expand-only '(solve)
    '(exists (x) (solve 5 (father 'pete x))))

(exists (x)
  (map (lambda (subst)
         (concretize-sequence (subst-in 'pete subst) (subst-in x subst)))
       (stream-prefix (- 5 1) (query (father 'pete x)))))
\end{alltt}

If we further expand \texttt{concretize-sequence}, we see
exactly what this does.

\begin{alltt}
(exists (x)
  (map 
    (lambda (subst)
      (let-values (ct new-env) (concretize-term (subst-in 'pete subst) '())
        (cons ct
          (let-values (ct new-env) (concretize-term (subst-in x subst) new-env)
            (cons ct '())))))
    (stream-prefix (- 5 1) (query (father 'pete x)))))
\end{alltt}

\texttt{Exercise 4:} Redefine \texttt{solve} to set up an interactive
loop to force more answers.  Use \texttt{0} to indicate no more
answers and use \texttt{+} to indicate more answers.

We can simplify things a bit when we want at most one result with
\texttt{solution},

\begin{alltt}
(define-syntax solution
  (syntax-rules ()
    [(_ x)
     (let ([ls (solve 1 x)])
       (if (null? ls) #f (car ls)))]))
\end{alltt}

\begin{alltt}
> (exists (x) (solution (father 'pete x)))

(pete sal)
\end{alltt}

\texttt{Exercise 5:} Rewrite \texttt{solution} by redefining
\texttt{initial-sk} and \texttt{initial-fk}.

Using \texttt{stream-prefix} meets our needs for this tutorial, but in
general, it may be too naive.  We may want all the answers until some
predicate holds about one or more of the answers.  For example, the
result might be a stream of integers that terminates after the third odd
integer appears.  In this case, it is impossible to know what integer bound
to pass to \texttt{stream-prefix}.  In those circumstances, it is best
to process the stream of substitutions using a hand-crafted procedure.
\newpage
\subsection{Sequences of antecedents}

In the definition of \texttt{child-of-male}, we have exactly one
antecedent.  We are going to look at relations with additional
antecedents.

Let's first redefine \texttt{father}.

\begin{alltt}
(define father
  (binary-extend-relation (a1 a2)
    (fact () 'john 'sam)
      (binary-extend-relation (a1 a2)
        (fact () 'sam 'pete)
        (binary-extend-relation (a1 a2)
          (fact () 'pete 'sal)
          (fact () 'pete 'pat)))))
\end{alltt}

Before going further, it is time to introduce
\texttt{extend-relation}, which takes one or more relations.

\begin{alltt}
(define-syntax extend-relation
  (syntax-rules ()
    [(_ (id ...) rel-exp) rel-exp]
    [(_ (id ...) rel-exp0 rel-exp1 ...)
     (binary-extend-relation (id ...) rel-exp0
       (extend-relation (id ...) rel-exp1 ...))]))
\end{alltt}

\noindent
Then we can redefine \texttt{father},

\begin{alltt}
(define father
  (extend-relation (a1 a2)
    (fact () 'john 'sam)
    (fact () 'sam 'pete)
    (fact () 'pete 'sal)
    (fact () 'pete 'pat)))
\end{alltt}

So, we see that Sam is the grandfather of Sal and Pat, and John is the
grandfather of Pete.  We might ask, ``Is Sam the grandfather of anyone
in our closed five-person world.''

\begin{alltt}
(define grandpa-sam
  (relation _ (grandchild)
    (to-show grandchild)
    (exists (parent)
      (all (father 'sam parent) (father parent grandchild)))))
\end{alltt}

\begin{alltt}
> (exists (y) (solve 6 (grandpa-sam y)))

((sal) (pat))
\end{alltt}

It is correct, because Sam is the father of Pete, and
Pete is the father of Sal and Pat.  Here is
how the substitutions that led to this answer were built.  First
\texttt{y} became some \texttt{grandchild}.  At this point, \texttt{y}
is not instantiated, only \emph{shared} with the variable
\texttt{grandchild}.  Then \texttt{parent} is instantiated to
Pete.  Next, \texttt{grandchild} is instantiated to
Sal, but we know that \texttt{y} is shared with
\texttt{grandchild}, so \texttt{y} is also instantiated to
Sal.  Then we fail, thus re-instantiating \texttt{grandchild}
to Pat.  Finally, the failures back all the way out, leading
to the empty list (the result of invoking \texttt{initial-fk}).

The macro \texttt{any} is the dual of \texttt{all}.  It succeeds
whenever one of its antecedents succeeds, whereas \texttt{all} fails
whenever one of its antecedents fails.  Our implementation uses
\texttt{extend-relation}.  Each antecedent is transformed into a zero
argument relation.  Then the result of creating the extended relation,
which we know to be a zero argument relation is, itself,
\emph{invoked}.

\begin{alltt}
(define-syntax any
  (syntax-rules ()
    [(_ ant ...) ((extend-relation () (relation _ () (to-show) ant) ...))]))
\end{alltt}

\texttt{Exercise 5.5:} Rewrite \texttt{(relation \_ () (to-show) ant)}
by expanding it and then observing what parts of the remainder can be
optimized away.  Finally, redefine \texttt{any} using the improved
code.

The implementation of the basic unsullied logic system is now
complete.  Most of what we may want to do with a logic system can be
programmed with this basic system comprised of \texttt{relation}
(and \texttt{fact}), \texttt{extend-relation} , \texttt{all},
\texttt{all!}, and \texttt{solve}.

Let's return (using \texttt{all}) to the most recent definition of
\texttt{grandpa-sam}.  Obviously this is not as general as we might
expect.  Consider an attempt to take us beyond concerns for Sam.

\begin{alltt}
(define grandpa-maker
  (lambda (grandad)
    (relation _ (grandchild)
      (to-show grandchild)
      (exists (parent)
        (all (father grandad parent) (father parent grandchild))))))

> (exists (x) (solve 6 ((grandpa-maker 'sam) x)))
\end{alltt}

\noindent
This just uses lexical scope to reconstruct \texttt{grandpa-sam}.
But, it still requires that we know who we want to find out about. We
do something a bit more abstract, below.  We postpone the
determination of the function \texttt{guide} by making it, too, a
variable. Here it is lexical, but it can be a logic variable, instead,
because procedures are treated as raw values.  Thus, we can pass the function
\texttt{father}, which then gets invoked.  This allows for the
potential creation of a more abstract relation.  For example, if we
had a \texttt{mother} definition, like our \texttt{father} definition,
then \texttt{grandpa-maker} would still work, as long as
\texttt{mother} is passed to \texttt{grandpa-maker} instead of
\texttt{father}.  Then we would have a matrilineal grandparent instead
of a patrilineal one.  Of course, then ``\texttt{grandpa-maker}'' and
``\texttt{grandad}'' would be poorly chosen names.

\begin{alltt}
(define grandpa-maker
  (lambda (guide grandad)
    (relation _ (grandchild)
      (to-show grandchild)
      (exists (parent)
        (all (guide grandad parent) (guide parent grandchild))))))

> (exists (x) (solve 4 ((grandpa-maker father 'sam) x)))
((sal) (pat))
\end{alltt}

Next we consider a more concrete problem, which is to use our logic
system to define \texttt{grandpa}.  Now, we can replace the lexical
variable \texttt{grandad} with the logic variable \texttt{grandad},
which leaves the decision completely open.  

Here is our first (somewhat naive) definition of \texttt{grandpa}.

\begin{alltt}
(define grandpa
  (relation _ (grandad grandchild)
    (to-show grandad grandchild)
    (exists (parent)
      (all (father grandad parent) (father parent grandchild)))))

> (exists (x) (solve 4 (grandpa 'sam x)))
((sam sal) (sam pat))
\end{alltt}

We make Pete an uncle of Betty and David, the children of his sister,
Polly.  First, we include Polly in the \texttt{father} relation.
Then, we define a \texttt{mother} relation, including some facts like
we did for the \texttt{father} relation.  Finally, we add a relation
to the \texttt{grandpa} relation, so that Pete's sister's children can
claim Sam as their grandfather.

\begin{alltt}
(define sam-polly (fact () 'sam 'polly))
(define father (extend-relation (a1 a2) father sam-polly))
\end{alltt}

\begin{alltt}
(define polly-betty (fact () 'polly 'betty))
(define polly-david (fact () 'polly 'david))
(define mother (extend-relation (a1 a2) polly-betty polly-david))
\end{alltt}

\begin{alltt}
(define grandpa
  (extend-relation (a1 a2) grandpa
    (relation _ (grandad grandchild)
      (to-show grandad grandchild)
      (exists (parent)
        (all (father grandad parent) (mother parent grandchild))))))
\end{alltt}

\begin{alltt}  
> (exists (y) (solve 10 (grandpa 'sam y)))

((sam sal) (sam pat) (sam betty) (sam david))
\end{alltt}

\noindent
And we discover that Sam is, indeed, the grandfather of
Betty and David.

\subsection{Lexically-shared variables}
In the previous definition of \texttt{grandpa}, both relations use the
same logic variables, so why should we write it twice?
Instead, we use \texttt{exists} like this,
\begin{alltt}
(define grandpa
  (exists (parent grandad grandchild)
    (extend-relation
      (relation _ ()
        (to-show grandad grandchild)
        (all (father grandad parent) (father parent grandchild)))
      (relation _ ()
        (to-show grandad grandchild)
        (all (father grandad parent) (mother parent grandchild))))))
\end{alltt}

And since the \texttt{to-show}'s are the same, we can use \texttt{any}
like this.

\begin{alltt}
(define grandpa
  (relation _ (grandad grandchild)
    (to-show grandad grandchild)
    (exists (parent)
      (any
        (all (father grandad parent) (father parent grandchild))
        (all (father grandad parent) (mother parent grandchild))))))
\end{alltt}

\texttt{Exercise 6:}
Determine if 
\begin{alltt}
(define grandpa
  (exists (parent)
    (relation _ (grandad grandchild)
    (to-show grandad grandchild)
      (any
        (all (father grandad parent) (father parent grandchild))
        (all (father grandad parent) (mother parent grandchild))))))
\end{alltt}
works and compare this variant with the previous ones.  Which one is
better?  There are relations for which this last style backfires.
Characterize when this placement of \texttt{exists} might backfire
by defining a more complicated relation.  (Althogh this problem can
be solved, here, it might be best to have understood the definition
of \texttt{ancestor}, below, before trying it.)
\newpage
\texttt{Exercise 7:} Suppose that the keyword \texttt{to-show} were
restricted to take a single argument.  Rewrite some variant of
\texttt{grandpa} with that restriction.  To get started, study the code
of \texttt{grandpa-sam}, below.

\begin{alltt}
(define grandpa-sam
  (let ([r (relation _ (child)
             (to-show child)
             (exists (parent)
               (all (father 'sam parent) (father parent child))))])
    (relation _ (child)
      (to-show child)
      (r child))))
\end{alltt}

In the remainder of the paper, we present some sullied operators,
consider the role of recursion in our logic system, and study some
famous examples that display some of the power of logic programming.

\section{Sullied antecedents}

In this section we introduce the sullied operators, which reveal some
parts of the underlying structure.  In each instance, we are expanding
the kinds of antecedents that are available.  Among these are the two
features that often work together: \emph{cut} and \texttt{fail}; four
operations for interfacing with Scheme: \texttt{pred-call},
\texttt{fun-call} \texttt{pred-nocheck}, and \texttt{fun-nocheck}; and
three useful miscellaneous operations: \texttt{fails},
\texttt{instantiated}, and \texttt{view-subst}.

The grammar for antecedents \texttt{<A>} completes the language.
\begin{alltt}
<A> = 
     <relation call> 
   | (all <A>*)
   | (all! (<cutter>) <A>*)
   | (any <A>*) 
   | <sullied antecedent>
   | <or any Scheme expression that evaluates to these,
       since these are all values.>

<sullied antecedent> = 
   | cut
   | fail
   | (pred-call <t> <t> ...)
   | (fun-call <t> <t> <t> ...)
   | (pred-call/no-check <t> <t> ...)
   | (fun-call/no-check <t> <t> <t> ...)
   | (fails <A>)
   | (instantiated <t>)
   | (view-subst <t>)
\end{alltt}

\noindent
where \texttt{<t>} is a term, and \texttt{<cutter>} should be a
lexical variable bound to the result of invoking \texttt{!!}
on a failure continuation.

\subsection{Short cuts and \texttt{fail}}

In this section we introduce the conventional \emph{cut} operator.
The \texttt{relation} macro takes a lexical variable as its first
argument, and it finds itself bound to \texttt{(!! cutk)} inside the
big \texttt{lambda} expression.  What \texttt{!!} does is take the
current \texttt{cutk} failure continuation and build an antecedent
that always succeeds, but as part of succeeding it installs that
\texttt{cutk} as the failure continuation.  Thus it exits the relation
call if failure backs into it.  This is far more drastic than
\texttt{tight-cut}, since it gives up on the whole call, not just the
relation perhaps within an extended relation.

We present three variants of \texttt{grandpa}, each with only one use
of \texttt{cut}, below. The first one places it as the last
antecedent of \texttt{grandpa/father}.  If we look
back at the definition of \texttt{relation}, we can see that
\texttt{cut} becomes a lexically-scoped variable.

\begin{alltt}
(define grandpa/father
  (relation cut (grandad grandchild)
    (to-show grandad grandchild)
    (exists (parent)
      (all (father grandad parent) (father parent grandchild) cut))))
\end{alltt}

\noindent
expands to

\begin{alltt}
(define grandpa/father
  (lambda (g g)
    (reify-cutters
      (lambda (tight-cut cut)
        (exists (grandad grandchild)
          (all (all! (tight-cut) (== grandad g) (== grandchild g))
            (exists (parent)
              (all (father grandad parent) (father parent grandchild) cut))))))))
\end{alltt}

The procedure \texttt{!!} takes a failure continuation \texttt{cutk}
and returns an antecedent.  If the antecedent is invoked, it
\emph{succeeds} and replaces the current failure continuation with
\texttt{cutk}.  Thus, if failure backs into it, the attempt to find a
result for this relation call is abandoned.  To see this in action,
we redefine \texttt{grandpa}, below.

\begin{alltt}
(define grandpa/mother
  (relation _ (grandad grandchild)
    (to-show grandad grandchild)
    (exists (parent)
      (all (father grandad parent) (mother parent grandchild)))))

(define grandpa
  (extend-relation (a1 a2) grandpa/father grandpa/mother))
\end{alltt}

\begin{alltt}
> (exists (x y) (solve 10 (grandpa x y)))

((john pete))
\end{alltt}

For \texttt{!!}, \texttt{exiting-fk} is the \texttt{cutk} of the
call to \texttt{grandpa}.  The effect of using \texttt{cut} at
the end is that once the first answer is found and failure forced, the
\texttt{exiting-fk} that has been passed as the failure continuation is
invoked, so it takes us completely out of the call to
\texttt{grandpa}.

Next, consider the following revision of \texttt{grandpa}, where
we swap the last two antecedents of \texttt{grandpa/father}.

\begin{alltt}
(define grandpa/father
  (relation cut (grandad grandchild)
    (to-show grandad grandchild)
    (exists (parent)
      (all (father grandad parent) cut (father parent grandchild)))))

(define grandpa
  (extend-relation (a1 a2) grandpa/father grandpa/mother))
\end{alltt}
\begin{alltt}
> (exists (x y) (solve 10 (grandpa x y)))

((john pete) (john polly))
\end{alltt}

\noindent 
The variable \texttt{grandad} cannot be re-instantiated because of the
invocation of the \texttt{exiting-fk}, but \texttt{parent} and
\texttt{grandchild} can. As failure works its way back into the same
call to \texttt{father}, new instantiations for \texttt{parent} and
\texttt{grandchild} are found.  If we run out of data to match
\texttt{parent} and \texttt{grandchild}, then instead of looking for
the next match of \texttt{grandad} and \texttt{parent}, we invoke the
\texttt{cutk} that was passed to \texttt{!!}. So, we have merely two
answers.  One way to describe this is to state that we only want the
\emph{first} father's grandchildren. So, if we moved the facts about
Pete and his children to the top of \texttt{father}, we would get no
answers, because then Pete would be the first father and he has no
grandchildren.

Finally, we have the last variation of \texttt{grandpa}, where the
\texttt{cut} is the first antecedent of \texttt{grandpa/father}.

\begin{alltt}
(define grandpa/father
  (relation cut (grandad grandchild)
    (to-show grandad grandchild)
    (exists (parent)
      (all cut (father grandad parent) (father parent grandchild)))))
\end{alltt}
\begin{alltt}
(define grandpa
  (extend-relation (a1 a2) grandpa/father grandpa/mother))
\end{alltt}

\begin{alltt}
> (exists (x y) (solve 10 (grandpa x y)))

((john pete) (john polly) (sam sal) (sam pat))
\end{alltt}

\noindent This variation yields only paternal grandfathers, because
when it runs out of fathers as potential grandfathers, \texttt{cutk}
is invoked.  This is the same as ignoring the second relation,
altogether.  If we reorganize the facts in \texttt{father} as we did
at the end of the previous example, we get the same four answers.  If
Pete were the father of Betty, too, then she would show up as Sam's
grandchild.  She would have appeared only once, however, because
\texttt{cutk}'s invocation has precluded us from using the
\texttt{grandpa/mother} relation in all three examples.

It is okay to use \texttt{cut} more than once within a single rule.
Be careful, however, to fully appreciate its consequences each time it
is used.  Such a powerful control mechanism must be handled with great
care.  In the type inferencer we have several antecedents where there
is more than one occurrence of \texttt{cut}.

An idiom is to cut and then fail immediately.  Doing the cut always
succeeds and the exiting doesn't happen until failure works its way
back into the cut.  One way to force that failure is with the
\texttt{fail} antecedent.  For example, we might say that if someone is a
\emph{mother} she cannot be a grand\emph{pa} under any circumstances.

\begin{alltt}
(define fail
  (lambda@ (sk fk subst cutk)
    (fk)))
\end{alltt}

\begin{alltt}
(define no-grandma
  (relation cut (grandad grandchild)
    (to-show grandad grandchild)
    (exists (parent)
      (all (mother grandad parent) cut fail))))

(define no-grandma-grandpa
  (extend-relation (a1 a2) no-grandma grandpa))

> (exists (x) (solution (no-grandma-grandpa 'polly x)))

#f
\end{alltt}        

\subsection{Interfacing Scheme functions}

Suppose we want to restrict the answers of \texttt{grandpa} so that
someone isn't a grandfather unless his child's name starts with the
letter, ``p.''  John's child's name is Sam, so John is no
longer considered a grandfather. But, Sam's child's name is
Pete, so Pete's children are still someone's grandchildren.
\newpage
\begin{alltt}
(define grandpa
  (relation _ (grandad grandchild)
    (to-show grandad grandchild)
    (exists (parent)
      (all 
        (father grandad parent) 
        (starts-with-p?-rel parent)
        (father parent grandchild)))))
\end{alltt}

\begin{alltt}
(define starts-with-p?
  (lambda (x)
    (and
      (symbol? x)
      (string=? (string (string-ref (symbol->string x) 0)) "p"))))

(define starts-with-p?-rel
  (lambda (x)
    (lambda@ (sk fk subst cutk)
      (if (starts-with-p? (nonvar! (subst-in x subst)))
        (@ sk fk subst cutk)
        (fk)))))

(define nonvar!
  (lambda (t)
    (if (var? t)
      (error 'nonvar! "Variable found in call: ~s" (concretize-var t))
      t)))
\end{alltt}

\begin{alltt}
> (exists (x y) (solve 10 (grandpa x y)))

((sam sal) (sam pat))
\end{alltt}

Let's look closely at the Scheme function \texttt{starts-with-p?-rel}.
Like other relations, it returns a function that expects the same
arguments: a success continuation, a substitution, and two failure
continuations.  If \texttt{x} (after it has been substituted for) is
needed for the actual test, then it should not still be a variable,
which is the job of \texttt{nonvar!}, above.  It is certainly possible
for an argument to get substituted by a variable, but in this case it
would not be meaningful. If \texttt{starts-with-p?} test is true, then
we take the success path, which is \emph{always} \texttt{(@ sk fk
subst cutk)}, otherwise we take the failure path, which is
\emph{always} \texttt{(fk)}.

The predicate \texttt{starts-with-p?} is arbitrary.  Any Scheme
predicate of any number of arguments can be used.  The Scheme
predicate acting as a relation can do anything that any Scheme
predicate can do, including capturing continuations, setting
variables, writing to files, etc.  The full panoply of options is
available to the user.  The only requirement is that these relations
take the same four arguments, and that they exit the same way:
\texttt{(@ sk fk subst cutk)} for success and \texttt{(fk)} for
failure. We abstract this with a new operator, \texttt{pred-call},
below.

\begin{alltt}
(define-syntax pred-call
  (syntax-rules ()
    [(_ p t ...)
     (lambda@ (sk fk subst cutk)
       (if ((nonvar! (subst-in p subst)) (nonvar! (subst-in t subst)) ...)
         (@ sk fk subst cutk)
         (fk)))]))
\end{alltt}

Also, we can call \emph{any} Scheme function and unify the results of
the call with something else.  For example, we can use
\texttt{(fun-call * 15 3 5)} as an antecedent.  We know that all but
the \emph{second} argument (\texttt{15}, here) should be a value after
substitution, so we are safe to apply the Scheme function to those
values.  We then unify the answer with the second argument.  If that
second argument is an uninstantiated variable or contains an
uninstantiated variable, then this causes the exiting substitution to
be larger than the entering one.

\begin{alltt}
(define-syntax fun-call
  (syntax-rules ()
    [(_ f t u ...)
     (lambda@ (sk fk subst cutk)
       (cond
         [(unify t ((nonvar! (subst-in f subst)) (nonvar! (subst-in u subst)) ...) subst)
          => (lambda (subst)
               (@ sk fk subst cutk))]
         [else (fk)]))]))
\end{alltt}

\begin{alltt}
> (exists (q) (solution (fun-call * q 3 5)))
(#<procedure *> 15 3 5)
\end{alltt}

The Scheme interface functions \texttt{pred-call} and
\texttt{fun-call} catch errors when calling in the presence of logic
variables.  There are times, however, when you want to pass a logic
variable to some Scheme procedure.  For thoes times, we use
\texttt{pred-call/no-check} and \texttt{fun-call/no-check}, below.
The \texttt{pred-call}s allow the user to treat Scheme
\texttt{pred}icates as logic predicates, whereas the
\texttt{fun-calls} allow the use to treat Scheme \texttt{fun}ctions as
logic functions (A logic function is a logic relation that has at most
one value for each input.)  Since functions return a value,
\texttt{fun-call} and \texttt{fun-call/no-check} cause that value to
get unified with the (possibly uninstantiated) first argument.  The
``\texttt{/no-check}'' versions, below, are for conveniently writing
extensions to the logic system that \emph{can} handle logic variables.

\begin{alltt}
(define-syntax pred-call/no-check
  (syntax-rules ()
    [(_ p t ...)
     (lambda@ (sk fk subst cutk)
       (if ((subst-in p subst) (subst-in t subst) ...)
         (@ sk fk subst cutk)
         (fk)))]))

(define-syntax fun-call/no-check
  (syntax-rules ()
    [(_ f t u ...)
     (lambda@ (sk fk subst cutk)
       (cond
         [(unify t ((subst-in f subst) (subst-in u subst) ...) subst)
          => (lambda (subst)
               (@ sk fk subst cutk))]
         [else (fk)]))]))
\end{alltt}

\subsubsection{\texttt{fails}, \texttt{instantiated}, and \texttt{view-subst}}

Below is the function \texttt{fails} that fails when its goal succeeds and
succeeds when its goal fails.  Basically, we hand build both the success
and failure continuations.

\begin{alltt}
(define fails
  (lambda (ant)
    (lambda@ (sk fk subst cutk)
      (@ ant
        (lambda@ (current-fk subst cutk) (fk))
        (lambda () (@ sk fk subst cutk))
        subst 
        cutk))))
\end{alltt}
\noindent
And here is an example of \texttt{fails}.

\begin{alltt}
(define grandpa
  (relation _ (grandad grandchild)
    (to-show grandad grandchild)
    (exists (parent)
      (all
        (father grandad parent)
        (fails (starts-with-p? parent))
        (father parent grandchild)))))
\end{alltt}

\begin{alltt}
> (exists (x y) (solve 10 (grandpa x y)))

((john pete) (john polly))
\end{alltt}

To determine if a variable, \texttt{t}, is instantiated, use
\texttt{(instantiated t)} as an antecedent and this definition.

\begin{alltt}
(define instantiated
  (lambda (t)
    (pred-call/no-check (lambda (x) (not (var? x))) t)))
\end{alltt}

\noindent
At this point, \texttt{t} is either some variable or some value.  

To view a substitution, use \texttt{(view-subst t)} with this
definition.  The definition of \texttt{concretize-subst} should
be obvious, since it simply concretizes every logic variable in
a substitution.  

\begin{alltt}
(define view-subst
  (lambda (t)
    (lambda@ (sk fk subst cutk)
      (pretty-print (subst-in t subst))
      (pretty-print (concretize-subst subst))
      (@ sk fk subst cutk))))

(define concretize-subst
  (letrec
    ([cs (lambda (subst env)
           (cond
             [(null? subst) '()]
             [else
               (let ([comm (car subst)])
                 (let-values (cv env) (concretize-var (commitment->var comm) env)
                   (let-values (ct env) (concretize-term (commitment->term comm) env)
                     (cons (commitment cv ct) (cs (cdr subst) env)))))]))])
    (lambda (subst)
      (cs subst '()))))
\end{alltt}

Here is a new definition of \texttt{grandpa}, like an earlier one,
but views the substitution.

\begin{alltt}
(define grandpa
  (relation _ (grandad grandchild)
    (to-show grandad grandchild)
    (exists (parent)
      (all
        (father grandad parent)
        (father parent grandchild)
        (view-subst grandchild)))))
\end{alltt}

\begin{alltt}
> (exists (x y) (solve 10 (grandpa x y)))

pete
((grandad.0 x.0)
 (grandchild.0 y.0)
 (x.0 john)
 (parent.0 sam)
 (y.0 pete))
\end{alltt}
\newpage
\begin{alltt}
polly
((grandad.0 x.0)
 (grandchild.0 y.0)
 (x.0 john)
 (parent.0 sam)
 (y.0 polly))
\end{alltt}

\begin{alltt}
sal
((grandad.0 x.0)
 (grandchild.0 y.0)
 (x.0 sam)
 (parent.0 pete)
 (y.0 sal))

pat
((grandad.0 x.0)
 (grandchild.0 y.0)
 (x.0 sam)
 (parent.0 pete)
 (y.0 pat))

((john pete) (john polly) (sam sal) (sam pat))
\end{alltt}

This completes the discussion of the features and implementation of
our logic system.  We have accomplished this using merely eight
people.  The population of our world is about to grow.

\section{Recursive definitions}

In this section, we introduce two interesting problems.  The first
finds the youngest common ancestor of two people in our world.  The
second is the well-known ``Towers of Hanoi'' problem.  The second 
is interesting because it uses recursion, \texttt{pred-call}, and
\texttt{fun-call} in one relation.  We start with the youngest common
ancestor problem.

\subsection{Youngest common ancestor}

Suppose that we want to know if someone is a (patrilineal) ancestor.
We know that if someone old is the father of someone young, then the
old person is a patrilineal ancestor.  But, also, if the old person is
the father of someone not so old, and the not so old person is the
ancestor of the young person, then we know that we have an ancestor.
This way of describing ancestor is defined directly in our logic
system.  We add a few more facts to \texttt{father} to make the
outcomes a bit more interesting.  Specifically, we add that ``John is
the father of Harry,'' ``Harry is the father of Carl,'' and ``Sam is the
father of Ed.''

\begin{alltt}
(define father
  (extend-relation (a1 a2) father
    (fact () 'john 'harry)
    (fact () 'harry 'carl)
    (fact () 'sam 'ed)))
\end{alltt}

We are now ready to solve the problem that we stated earlier.  To do
this, we first define and test the ancestor relation.

\begin{alltt}
(define ancestor
  (extend-relation
    (relation _ (old young)
      (to-show old young)
      (father old young))
    (relation _ (old young)
      (to-show old young)
      (exists (not-so-old)
        (all (father old not-so-old) (ancestor not-so-old young))))))
\end{alltt}

\texttt{Exercise 8:} Redefine the previous definition of
\texttt{ancestor} as one relation using using \texttt{any}.

Here is a simple test.

\begin{alltt}
> (exists (x y) (solve 100 (ancestor x y)))

((john sam)
 (sam pete)
 (pete sal)
 (pete pat)
 (sam polly)
 (john harry)
 (harry carl)
 (sam ed)
 (john pete)
 (john polly)
 (john ed)
 (john sal)
 (john pat)
 (sam sal)
 (sam pat)
 (john carl))
\end{alltt}

Once we have the concept of ancestor, it is easy to think in terms of
a common ancestor.  Two people share a common ancestor if somewhere
along their respective ancestor chains, their paths cross.  Here is
how we can write that in our logic system.
\newpage
\begin{alltt}
(define common-ancestor
  (relation _ (young-a young-b old)
    (to-show young-a young-b old)
    (all (ancestor old young-a) (ancestor old young-b))))

> (exists (x) (solve 4 (common-ancestor 'pat 'ed x)))

((pat ed john) (pat ed sam))
\end{alltt}

\noindent
This says that John and Sam are both common
ancestors of Pat and Ed.

If two people share two common ancestors, then we can determine if one
of the common ancestors is younger than the other common ancestor.

\begin{alltt}
(define younger-common-ancestor
  (relation _ (young-a young-b old not-so-old)
    (to-show young-a young-b old not-so-old)
    (all
      (common-ancestor young-a young-b not-so-old)
      (common-ancestor young-a young-b old)
      (ancestor old not-so-old))))
\end{alltt}

\noindent
According to \texttt{relation}, the ``\texttt{(all}'' and matching
``\texttt{)}'' are unnecessary, but we use then for clarification.

\begin{alltt}
> (exists (x y) (solve 4 (younger-common-ancestor 'pat 'ed x y)))

((pat ed john sam))
\end{alltt}

We finally come to the problem that we are most interested in, which
is how do we determine the youngest common ancestor.  We already know
that Pat and Ed share Sam and John, but we want the youngest among the
common ancestors. Since John must be older than Sam, the answer should
be Sam.

\begin{alltt}
(define youngest-common-ancestor
  (relation _ (young-a young-b not-so-old)
    (to-show young-a young-b not-so-old)
    (all
      (common-ancestor young-a young-b not-so-old)
      (exists (y)
        (fails (younger-common-ancestor young-a young-b not-so-old y))))))

> (exists (x) (solve 4 (youngest-common-ancestor 'pat 'ed x)))
((pat ed sam))
\end{alltt}

\noindent
The tricky part of this is that if we find a younger ancestor,
then the one we have chosen is not the youngest common ancestor.  So, the
ancestor is the youngest common ancestor provided each attempt to find
a younger common ancestor fails.

What is interesting about this approach is that it is recursive: we
define \texttt{ancestor} in terms of \texttt{ancestor}.  Although
we don't use recursion to implement our logic system, our model
relies heavily on the idea that users are facile with recursion
and we rely heavily on the fact that Scheme supports recursion.  For
instance, the recursion supported by \texttt{define} in these four
definitions could just as easily have been supported using
\texttt{letrec}.  Furthermore, we can bind \texttt{father} lexically,
so that the expression takes any binary relation as an argument.

\begin{alltt}
(define youngest-common-ancestor
  (lambda (father)
    (letrec ([ancestor ...]
             [common-ancestor ...]
             [younger-common-ancestor ...]
             [youngest-common-ancestor ...])
      youngest-common-ancestor)))
\end{alltt}  

\noindent
This would also solve a problem related to an organizational chart.
If we pass a relation, say \texttt{supervisor}, instead of
\texttt{father}, then \texttt{youngest-common-ancestor} would find the
least common supervisor.  In general, this procedure produces the
\emph{least upper bound}, provided one exists, of a \emph{finite}
binary relation.

\subsection{Comparison with Seres and Spivey}
The similarities with Seres and Spivey are striking.  In fact, a
subset of our model maps directly onto theirs, although our goal
has been a full implementation of logic programming, including
many of the sullied operations as well as some meta operations.

Using \texttt{==} our Scheme embedding resembles their Haskell
embedding.  Where we differ is that we would be limited to using only
\texttt{any}, \texttt{all}, and \texttt{==}.  We would no longer have
\texttt{relation} and \texttt{extend-relation}, and all the sullied
operators.  This means that we could use \texttt{lambda} to define
relations.  For example, we can define \texttt{father} and
\texttt{ancestor} like this,

\begin{alltt}
(define father
  (lambda (dad child)
    (any
      (all (== dad 'john) (== child 'sam))
      (all (== dad 'sam) (== child 'pete))
      (all (== dad 'sam) (== child 'polly))
      (all (== dad 'pete) (== child 'sal))
      (all (== dad 'pete) (== child 'pat))
      (all (== dad 'john) (== child 'harry))
      (all (== dad 'harry) (== child 'carl))
      (all (== dad 'sam) (== child 'ed)))))
\end{alltt}
\newpage
\begin{alltt}
(define ancestor
  (lambda (old young)
    (any
      (father old young)
      (exists (not-so-old)
        (all (father old not-so-old) (ancestor not-so-old young))))))
\end{alltt}

\noindent
And this can be tested similarly.
\begin{alltt}
> (exists (x) (solve 20 (ancestor 'john x)))

((john sam)
 (john harry)
 (john pete)
 (john polly)
 (john ed)
 (john sal)
 (john pat)
 (john carl))
\end{alltt}

Does the Haskell embedding differ in any significant way from our
embedding in Scheme?  Yes, besides avoiding sullied operators, there
is a fundamental difference.  Everything that works in the Haskell
embedding works in the Scheme embedding, but not vice versa.  Because
the Haskell embedding does unification piecewise within an
\texttt{all}, any time that the unification fails to match, there is
automatic backtracking.  That is not the case with \texttt{relation}
and \texttt{fact}.  In our embedding, if unification fails, it invokes
the failure continuation, immediately.  It is as though each relation
has a lock on it and a term either opens it or a new one is tried.  In
order to accomplish this in the Haskell embedding, there would need to
be \texttt{(all! (tight-cut) ...)}, or something similar.

In sum, the Haskell embedding is good as far as it goes, but it
leaves many of the more interesting aspects of logic programming
unresolved.  The approach of treating every antecedent as a stream
of substitutions is not far from our approach.  Since we have
substitutions \texttt{subst} and failure continuations \texttt{fk}
in every antecedent, we could use a different representation of the
body of the antecedent, where all but \texttt{fk} would comprise
a data structure and the \texttt{fk} would be a stream (thunk).
Then each antecedent closure could be modelled as the \texttt{cons}ing
of the data structure to a stream.  Because this is a relatively
small program, making this last step in order to use the stream
monad seems like overkill, especially since this approach does not
appear to extend naturally to the all important sullied antecedents.

\subsection{The difference between success and truth}

We revisit \texttt{any} to make a point about the difference between
\emph{success} and \emph{truth}.  Consider the following three
relations (\texttt{test1}, \texttt{test2}, and \texttt{test3}), which
rely only on \texttt{lambda} as in the Haskell embedding.

\begin{alltt}
> (define test1
    (lambda (x)
      (any (pred-call < 4 5) (fun-call < x 6 7))))

> (exists (x) (solution (test1 x)))

(x.0)
\end{alltt}
\begin{alltt}
> (define test2
    (lambda (x)
      (any (pred-call < 5 4) (fun-call < x 6 7))))

> (exists (x) (solution (test2 x)))

(#t)
\end{alltt}
\begin{alltt}
> (define test3
    (lambda (x y)
      (any (fun-call < x 5 4) (fun-call < y 6 7))))

> (exists (x y) (solution (test3 x y)))

(#f y.0)
\end{alltt}

The first two should not be much of a surprise.  In the first example,
\texttt{4} is less than \texttt{5}, so \texttt{x} remains uninstantiated,
and \texttt{6} is less than \texttt{7}, so \texttt{x} is instantiated in
the second example.  In the last example, however, we discover that
\texttt{y} remains uninstantiated even though \texttt{5} is \emph{not} less
than \texttt{4}.  Why? A \texttt{fun-call} always succeeds if its second
argument is an uninstantiated variable.  So, succeeding is not the same as
being true in a logic system.

\subsection{``Towers of Hanoi''}

Three poles (a \emph{left}, a \emph{middle}, and a \emph{right}) can hold
disks of various sizes provided that a larger one is never on top of a
smaller one.  The initial state of the problem has a set of \emph{n}
disks (all different sizes) sitting on the left pole.  The goal is to
place the entire set of disks on the middle pole.  Only the top disk
of any pole can be moved to a different pole and then that disk
becomes the top disk of the chosen pole.\footnote{This solution has been derived from
page 141 of \emph{Programming in Prolog Fourth Edition} by
W.\ F.\ Clocksin and C.\ S.\ Mellish.}
\newpage
\begin{alltt}
(define towers-of-hanoi
  (letrec
      ([move
         (extend-relation
           (relation cut ()
             (to-show 0 _ _ _)
             (all cut))
           (relation _ (n a b c)
             (to-show n a b c)
             (exists (m)
               (all
                 (fun-call - m n 1)
                 (move m a c b)
                 (pred-call printf "Move a disk from ~s to ~s~n" a b)
                 (move m c b a)))))])
    (relation _ (n)
      (to-show n)
      (move n 'left 'middle 'right))))
\end{alltt}

\begin{alltt}
> (begin (solution (towers-of-hanoi 3)) (void))

Move a disk from left to middle
Move a disk from left to right
Move a disk from middle to right
Move a disk from left to middle
Move a disk from right to left
Move a disk from right to middle
Move a disk from left to middle
\end{alltt}

The algorithm is straightforward. First, figure out how to solve the
problem for one fewer disks and then move that stack of disks as a
virtual disk.  The use of \texttt{pred-call} allows us to have a
procedure that writes one move.  Since we are displaying information,
we know that the value of \texttt{printf} is not false.  Also, the
\texttt{move} subtracts one from the number of disks using
\texttt{fun-call} and its first relation uses the anonymous variable
three times.  (Recall that the anonymous variable unifies with
everything, but it adds nothing to the substitution; the underscore
after \texttt{relation} has been chosen arbitrarily.  If the scope
uses undersore, choose a variable that is not live in the scope.  You
can always just use \texttt{cut}, however limiting its use to when the
program relies on it seems wiser.

\section{Three famous problems}

Three famous problems that we discuss are the so-called
\texttt{append} problem, the type-inference problem of a typed variant
of the lambda calculus with constants, conditional expressions,
primitives, and polymorphic \texttt{let}, and a generalization of
Prolog's \texttt{name} pedicate.  Before we can begin this walk down
memory lane, we must enlarge the set of possible terms and
consequently change the behavior of three functions:
\texttt{subst-in}, \texttt{concretize-term}, and \texttt{unify*}.

\subsection{Enlarging the set of terms}

Presently, a term is a variable or something that can be compared
trivially.  Now, we include in the set of terms those values that
cannot be compared trivially, such as pairs and vectors.  We restrict
our concerns to pairs.

\begin{alltt}
(define subst-in
  (lambda (t subst)
    (cond
      [(var? t)
       (cond
         [(assv t subst) => commitment->term]
         [else t])]
      [(pair? t) (cons (subst-in (car t) subst) (subst-in (cdr t) subst))]
      [else t])))

(define concretize-term
  (lambda (t env)
    (cond
      [(var? t) (concretize-var t env)]
      [(pair? t)
       (let-values (carct env) (concretize-term (car t) env)
         (let-values (cdrct env) (concretize-term (cdr t) env)
           (values (cons carct cdrct) env)))]
      [else (values t env)])))
\end{alltt}

These are not significant changes.  They each say walk recursively
through the term and replace all variables by their association in a
substitution (or in a threaded environment in the case of \texttt{concretize-term}).
\newpage
Here is an interesting example,

\begin{alltt} 
> (exists (x y z)
    (let ([term `(p ,x ,y (g ,z))])
      (let ([s (compose-subst (unit-subst y z) (unit-subst x `(f ,y)))]
            [r (compose-subst (unit-subst x 'a) (unit-subst z 'b))])
        (let ([new-term (subst-in term s)])
          (printf "~s~n" (concretize new-term))
          (printf "~s~n" (concretize (subst-in new-term r)))
          (let ([sr (compose-subst s r)])
            (printf "~s~n" (concretize-subst sr))
            (concretize (subst-in term sr)))))))
\end{alltt}

\begin{alltt}
(p (f y.0) z.0 (g z.0))
(p (f y.0) b (g b))
((y.0 b) (x.0 (f y.0)) (z.0 b))
(p (f y.0) b (g b))
\end{alltt}

In the example, we demonstrate a fact about substitutions: it does not
matter if we apply the substitutions to a term one at a time or apply
the composed substitution to the same term.

Next, we must change \texttt{unify*} (Recall that \texttt{unify*} is
called from \texttt{unify}.) to accomodate this new kind of term.  We
can ask if two data structures are \texttt{equal?}, which does a
recursive tree walk, comparing subparts.  If they are equal,
\texttt{equal?} responds with true, otherwise, it responds with false.
The redefinition of \texttt{unify*}, below, shares that attribute with
\texttt{equal?}. That is, when two data structures are equal,
\texttt{unify*} returns the empty substitution, otherwise, it returns
false.  But, unification is more than just equality.  The function
\texttt{unify*} takes two terms, and returns a substitution that allows
for the two terms to be perceived as \emph{equal} if the substitution
were applied to the two terms.  To accomplish this, each variable in
the two terms is added to the substitution by associating some term
with it.  In the recursive tree walk, two leaves that are aligned must
be \texttt{equal?}.  If one argument has a variable and the other one
contains something other than a variable, then that pair is added to
the substitution.  If the same variable is aligned in both arguments,
then the substitution remains unchanged.  If both are variables, then
one of them is treated as the term.  Each time something is added to
the substitution, it is a commitment. So, as the recursive tree walk
continues, it refines previous commitments.
\newpage
The version of \texttt{unify*}, below, takes any two terms and returns a
substitution if the terms unify and returns false, otherwise.

\begin{alltt}
(define unify*
  (lambda (t u)
    (cond
      [(trivially-equal? t u) empty-subst]
      [(var? t) (if (occurs? t u) #f (unit-subst t u))]
      [(var? u) (if (occurs? u t) #f (unit-subst u t))]
      [(and (pair? t) (pair? u))
       (cond
         [(unify* (car t) (car u))
           => (lambda (s-car)
                (cond
                  [(unify* (subst-in (cdr t) s-car) (subst-in (cdr u) s-car))
                   => (lambda (s-cdr)
                        (compose-subst s-car s-cdr))]
                  [else #f]))]
         [else #f])]
      [else #f])))
\end{alltt}
There are only three kinds of terms for us to consider: trivial
values, variables, and pairs.  The first case is straightforward.  Two
terms (including two variables) unify and return the empty
substitution if they are trivially equal.  If at least one of them is
a variable, then the other becomes the term associated with it,
provided that that term does not contain an occurrence of the
variable.  Two pairs might unify if their \texttt{car}s unify.  But,
before we determine if the \texttt{cdr}s unify, we must take the
substitution returned from successfully unifying the \texttt{car}s,
its commitments, and apply it to the \texttt{cdr}s.  This brings the
\texttt{cdr}s up-to-date with those commitments.  Then all the
commitments from unifying the \texttt{car}s, \texttt{s-car}, and all
the commitments from unifying the substituted for \texttt{cdr}s,
\texttt{s-cdr}, are composed.

Do not underestimate just how difficult it is to understand the pair
case.  As it it written, it looks rather straightforward, but when we
process a deeply nested term, we make lots of commitments and they
must be repeatedly refined by future substitutions.  It is a testament
to the power of recursion that it can look this simple.

The check for an occurrence of a variable in a term, below, is a
straightforward recursive tree walk, however, for reasons of efficiency,
most logic systems have chosen not to include it.  Later, we show a
variant of unification that simply avoids it even though there might
be an occurrence.

\begin{alltt}
(define occurs?
  (lambda (var t)
    (cond
      [(var? t) (eqv? t var)]
      [(pair? t) (or (occurs? var (car t)) (occurs? var (cdr t)))]
      [else #f])))
\end{alltt}

\texttt{Exercise 9:} Each of the functions, \texttt{subst-in},
\texttt{concretize-term}, \texttt{unify*}, and \texttt{occurs?}
treats pairs.  Include vectors as terms and allow the values in
a vector to be any term.

\texttt{Exercise 10:} All calls to \texttt{unify} can be improved by
passing it success and failure continuations, making sure to thread
them through the auxiliary calls.  Revise the entire system to make
this improvement.

\subsection{Unification examples}

There are many ways for two terms to fail to unify.  For example, it
is possible for the same variable to appear twice in one structure but
be associated with different integers.  Another easy way for two terms
to fail to unify is to have the same variable appear once in each term
but be associated with different integers.  These two ways are
demonstrated below. We start with some handy (global) logic variables,

\begin{alltt}
> (define u (exists (u) u))
> (define w (exists (w) w))
> (define x (exists (x) x))
> (define y (exists (y) y))
> (define z (exists (z) z))
\end{alltt}

\begin{alltt}
> (unify `(,x ,4) `(3 ,x) empty-subst)

#f

> (unify `(,x ,x) '(3 4) empty-subst)

#f
\end{alltt}

\noindent
In the preceding examples, \texttt{x} commits to \texttt{3},
but then looking at the second elements of each term, there is an
attempt to commit \texttt{x} to \texttt{4}, which violates an
earlier commitment.  The following two examples do unify, however.

\begin{alltt}
> (concretize-subst (unify `(,x ,y) '(3 4) empty-subst))

((x.0 3) (y.0 4))

> (concretize-subst (unify `(,x 4) `(3 ,y) empty-subst))

((x.0 3) (y.0 4))
\end{alltt}

\noindent
First, \texttt{x} commits to \texttt{3} and then \texttt{y}
commits to \texttt{4}.  Let's try a slightly more difficult
example.

\begin{alltt}
> (concretize-subst (unify `(,x 4 3 ,w) `(3 ,y ,x ,z) empty-subst))

((x.0 3) (y.0 4) (w.0 z.0))
\end{alltt}

\noindent
We can see that the last two items in each term do not violate earlier
commitments.  Thus, the two terms unify.  If any other integer appears
where the \texttt{3} in the first argument appears, then these fail to
unify.  If \texttt{y} appears where the \texttt{x} appears in the
second argument, then these also fail to unify.  The variable
\texttt{w} shares with the variable \texttt{z}, which means that when
one of them commits to some non-variable term, so does the other
one. Consider these two examples.

\begin{alltt}
> (concretize-subst (unify `(,x 4) `(,y ,y) empty-subst))

((x.0 4) (y.0 4))
\end{alltt}

\begin{alltt}
> (unify `(,x 4 3) `(,y ,y ,x) empty-subst)

#f
\end{alltt}
\noindent

\noindent
In the first example, \texttt{x} commits to \texttt{y} and agrees to
be associated with the same term.  Then \texttt{y} commits to
\texttt{4}.  In the second example, once \texttt{y} has committed to
\texttt{4}, since \texttt{y} and \texttt{x} are shared, \texttt{x} is
committed to \texttt{4}, but \texttt{x} tries to commit to \texttt{3},
which violates an earlier commitment.

\subsubsection{More unification examples}

We present five unification examples, below.

\begin{alltt}
> (concretize-subst (unify `(,w (,x (,y ,z) 8)) `(,w (,u (abc ,u) ,z)) empty-subst))

((x.0 8) (y.0 abc) (z.0 8) (u.0 8))
\end{alltt}

\begin{alltt}
> (concretize-subst (unify `(p (f a) (g ,x)) `(p ,x ,y) empty-subst))

((x.0 (f a)) (y.0 (g (f a))))
\end{alltt}

\begin{alltt}
> (concretize-subst (unify `(p (g ,x) (f a)) `(p ,y ,x) empty-subst))

((y.0 (g (f a))) (x.0 (f a)))
\end{alltt}

\begin{alltt}
> (concretize-subst (unify `(p a ,x (h (g ,z))) `(p ,z (h ,y) (h ,y)) empty-subst))

((z.0 a) (x.0 (h (g a))) (y.0 (g a)))
\end{alltt}

\begin{alltt}
> (unify `(p ,x ,x) `(p ,y (f ,y)) empty-subst)

#f
\end{alltt}

\noindent
The first example commits \texttt{w} to itself, which is no commitment
at all.  Then \texttt{x} shares with \texttt{u}, \texttt{y} commits to
the symbol \texttt{abc}, \texttt{z} shares with \texttt{u}, and
\texttt{z} commits to \texttt{8}.  In the second example, the
\texttt{p} symbols match, leading to no commitments.  Then, \texttt{x}
commits to \texttt{(f a)}.  Finally, \texttt{y} commits to \texttt{(g
(f a))}, since the \texttt{x} in \texttt{`(g ,x)} gets replaced by
what \texttt{x} is committed to, \texttt{(f a)}.  The third example
produces the same values for the variables, but in a different way.
Again, \texttt{p} symbols lead to no commitments.  Then \texttt{y}
commits to \texttt{`(g ,x)}. Finally, \texttt{x} commits to \texttt{(f
a)}.  The fourth example first commits \texttt{z} to the symbol
\texttt{a}.  Then, \texttt{x} commits to \texttt{`(h ,y)}. Next the
\texttt{h} symbols contribute nothing.  Finally, \texttt{y} commits to
\texttt{`(g ,z)}.  But, since \texttt{z} has committed to \texttt{a},
we know that \texttt{y} must become \texttt{(g a)}, which is why
\texttt{x}'s value is \texttt{(h (g a))}.  The fifth example fails to
unify, but only when we reach the last item in each term.  The
\texttt{p} symbols unify.  Then, \texttt{x} is shared with \texttt{y}.
When we try to unify \texttt{x} with \texttt{`(f ,y)}, we discover
that since \texttt{y} has been replaced by \texttt{x}, we are really
unifying \texttt{x} with \texttt{(f x)}.  It should be clear that this
is a violation of the \emph{occurs check} in the second \texttt{cond}
clause of \texttt{unify}.

\subsubsection{A lazy \texttt{unify}}

A different approach to writing \texttt{unify} is to postpone when we
apply a substitution.  This postponing makes it possible to avoid
applying substitutions in two separate circumstances.  First, if
somewhere deep in the two terms we have not seen any variable, but we
have found a place where the two terms fail to unify, then we return
false without having applied any substitutions.  Also, if the two
terms unify, but contain no variables, we also avoid applying a
substitution.  This differs significantly from the preceding
\texttt{unify*}, where we applied a substitution to every
\texttt{cdr}.

In this definition of \texttt{unify}
\footnote{This version is similar to the algorithm on pages 219--222 in Dybvig's
\emph{The Scheme Programming Language ANSI Scheme, second edition}. There,
however, the substitutions are represented as functions from terms to
terms.}, below, the excitement happens only when a variable is found in at
least one of the terms.  Otherwise, we walk recursively through the two
terms making sure that they are equal until we find a variable in either
term. Then we pass the variable, the substituted for other term, and
the substitution to \texttt{unify-with-var}, below.

\begin{alltt}
(define unify
  (lambda (t u subst)
    (cond
      [(trivially-equal? t u) subst]
      [(var? t) (unify/var t (subst-in u subst) subst)]
      [(var? u) (unify/var u (subst-in t subst) subst)]
      [(and (pair? t) (pair? u))
       (cond
         [(unify (car t) (car u) subst)
          => (lambda (subst)
               (unify (cdr t) (cdr u) subst))]
         [else #f])]
      [else #f])))
\end{alltt}
\newpage
\begin{alltt}
(define unify/var
  (lambda (t-var u subst)
    (cond
      [(assv t-var subst) (unify (subst-in t-var subst) u subst)]
      [(occurs? t-var u) #f]
      [else (compose-subst subst (unit-subst t-var u))])))
\end{alltt}

In \texttt{unify/var}, we must compose the substitution with a
commitment to be built with the variable \texttt{t-var} and the term
\texttt{u}.  As part of the call, we have brought \texttt{u}
up-to-date.  If \texttt{t-var} is bound in the substitution, we start
the whole process over again after bringing \texttt{t-var} up-to-date,
too.  Otherwise, we try to commit \texttt{u} to \texttt{t-var}.

\texttt{Exercise 11:} Determine the value of the unifications examples
given above.  Which ones produce the same answers?  Of the ones that
produce different answers, how close are they to the earlier answers.
Write a program that transforms substitutions yielded by the lazy
unifier and produces the same substitutions yielded by the earlier
one.
\subsection{Unification without the occurs check}

It is also possible to completely avoid the occurs check even if
there is an occurs-check violation.  In most Prologs, there is a
conscious decision to avoid the occurs check, but not too many of
them have a way to avoid it even if there is a violation. In
this section we present such a unifier.\footnote{The trick used here
also appears in Quintus Prolog.}

Before we can discuss the unifier, we must redefine \texttt{subst-in}
to account for decisions that will be made within the unifier.  We
choose to use \texttt{subst-in} sparingly, so when we do use it, we
have more work to do.  Before, \texttt{subst-in} merely replaced each
variable by its its associated term.  Now, however, it starts over
with that term.  If it is not bound, it stops.

\begin{alltt}
(define subst-in
  (lambda (t subst)
    (cond
      [(var? t)
       (cond
         [(assv t subst)
          => (lambda (c)
               (subst-in (commitment->term c) subst))]
         [else t])]
      [(pair? t)
       (cons (subst-in (car t) subst) (subst-in (cdr t) subst))]
      [else t])))
\end{alltt}

The unifier below avoids \texttt{subst-in} until it is ready to do
a \texttt{compose-subst}.  The two auxiliary procedures each take
a logic variable, but one takes another logic variable and the other
takes something that is not a logic variable.  

\begin{alltt}
(define unify
  (lambda (t u subst)
    (cond
      [(trivially-equal? t u) subst]
      [(var? t) (if (var? u) (unify-var/var t u subst) (unify-var/value t u subst))]
      [(var? u) (unify-var/value u t subst)]
      [(and (pair? t) (pair? u))
       (cond
         [(unify (car t) (car u) subst)
          => (lambda (car-subst)
               (unify (cdr t) (cdr u) car-subst))]
         [else #f])]
      [else #f])))
\end{alltt}

The procedure \texttt{unify-var/var}, unifies with the term associated
withw the first logic variable against the term of the second one, if
they are bound.  If one of them is bound and the other is not, then we
set up a loop in \texttt{unbound/bound}, below, that follows the bound
one until either the same variable comes up, or it ends up with a
value.  If both variables are unbound, and we already know that they
are not the same from the test in \texttt{unify}, then we naively
extend the substitution.  It's probably easier to skip to the bound
\texttt{u-var} case first, before studying the bound \texttt{t-var}
case.

\begin{alltt}
(define unify-var/var
  (lambda (t-var u-var s)
    (cond
      [(assv t-var s)
       => (lambda (ct)
            (cond
              [(assv u-var s)
               => (lambda (cu)
                    (let ([u-term (commitment->term cu)]
                          [t-term (commitment->term ct)])
                      (unify t-term u-term s)))]
              [else ((unbound/bound u-var s) ct)]))]
      [(assv u-var s) => (unbound/bound t-var s)]
      [else (extend-subst t-var u-var s)])))
\end{alltt}
\newpage
\begin{alltt}
(define unbound/bound
  (lambda (t-var s)
    (lambda (cu)
      (let loop ([cm cu])
        (let ([u-term (commitment->term cm)])
          (cond
            [(eqv? u-term t-var) s]
            [(var? u-term)
             (cond
               [(assv u-term s) => loop]
               [else (unify-var/value t-var u-term s)])]
            [else (extend-subst t-var u-term s)]))))))
\end{alltt}
The procedure \texttt{unify-var/value} is the interesting
part of this algorithm.  Now we know that the second argument
is a non-variable (i.e., a value).  First, we check to see
if the variable is bound and if so, unify against its associated term..
Now, if the variable is \emph{unbound} and the value of the
second argument is a \emph{pair}, then we have the situation
like before when we used the occurs check.  But, this time,
we will do something else.  Since we know that we have a pair,
we construct a pair with two new logic variables.  Then we
place the pair in the substitution.  We use the new substitution
to unify the first new logic variable with the \texttt{car} of
the term and when that returns, we use the resultant substitution
to unify the second new logic variable with the \texttt{cdr} of
the term.  If it not a pair, we do the same as above.

\begin{alltt}
(define unify-var/value
  (lambda (t-var u-value s)
    (cond
      [(assv t-var s)
       => (lambda (ct)
            (let ([t-term (commitment->term ct)]) 
              (unify t-term u-value s)))]
      [(pair? u-value)
       (let ([car-var (var ':a)]
             [cdr-var (var ':d)])
         (cond
           [(unify car-var (car u-value)
              (extend-subst t-var (cons car-var cdr-var) s))
            => (lambda (s)
                 (unify cdr-var (cdr u-value) s))]
           [else #f]))]
      [else (extend-subst t-var u-value s)])))
\end{alltt}
This allows us to produce a legitimate substitution in the example
that failed the occurs check.  In order to actually view the
substitution, we need to use an auxiliary procedure
\texttt{subst-vars-recursively}, which is nearly the same as
\texttt{subst-in}, but when it follows the associated term of a
variable, it makes sure that the associated commitment is not found
again by removing it from the substitution.

\begin{alltt}
(define subst-vars-recursively
  (lambda (t subst)
    (cond
      [(var? t)
       (cond
         [(assv t subst) =>
          (lambda (c)
            (subst-vars-recursively
              (commitment->term c) (remq c subst)))]
         [else t])]
      [(pair? t)
       (cons
         (subst-vars-recursively (car t) subst)
         (subst-vars-recursively (cdr t) subst))]
      [else t])))

> (concretize-subst
    (let ([s (unify `(p ,x ,x) `(p ,y (f ,y)) empty-subst)])
      (let ([vars (map commitment->var s)])
        (map commitment vars (subst-vars-recursively vars s)))))

((:d.0 ())
 (:a.0 (f :a.0))
 (:d.1 ((f . :d.1)))
 (:a.1 f)
 (y.0 (f (f . :d.1)))
 (x.0 (f (f . :d.1))))
\end{alltt}

If we look closely at the output, we can see the circularity in the
associated term of \texttt{:a.0} and \texttt{:d.1}.  Therefore, any
term that contains them is circular, which includes the terms
associated with \texttt{x.0}, and \texttt{y.0}.

\subsubsection{``Towers of Hanoi'' revisited}

Before, we look at the three famous problems, let's take another look at
the ``Towers of Hanoi'' problem.  It is less than satisfying that the
solution is not a value returned but just some displaying of information.
Instead, we can replace those effects by other effects and build
the answer in a table.  Then, that path can be the result.  Thus, we are
free to write functions that process the path.  For example, we can now
determine how many steps it takes for any n.  Before, we were limited by
our willingness to read and process screens or files.
\newpage
\begin{alltt}
(define towers-of-hanoi-path
  (let ([steps '()])
    (let ([push-step (lambda (x y) (set! steps (cons `(,x ,y) steps)))])
      (letrec
        ([move
           (extend-relation
             (relation cut ()
               (to-show 0 _ _ _)
               (all cut))
             (relation _ (n a b c)
               (to-show n a b c)
               (exists (m)
                 (all
                   (fun-call - m n 1)
                   (move m a c b)
                   (pred-call push-step a b)
                   (move m c b a)))))])
        (relation _ (n path)
          (to-show n path)
          (begin
            (set! steps '())
            (any
              (fails (move n 'l 'm 'r))
              (== path (reverse steps)))))))))
\end{alltt}

\begin{alltt}
> (exists (path) (solution (towers-of-hanoi-path 3 path)))

(3 ((l m) (l r) (m r) (l m) (r l) (r m) (l m)))
\end{alltt}

The primary difference between this version and the earlier version is that
in this version there is a lexical variable \texttt{steps} that holds each
step, where before we printed each step.  Then, by forcing failure with
\texttt{fails}, we are guaranteed to process the second rule.  It always
succeeds, since \texttt{path} is guaranteed to be uninstantiated. We
reverse the steps so that it looks like our earlier output. Everything else
is the same.  

\texttt{Exercise 12:} Use this definition of
\texttt{towers-of-hanoi-path} to produce a table of the number of
disks with the number of steps it takes to move that number of disks.

Our unifier now handles pairs (lists), so we can continue the
discussion of the three famous problems.
\newpage
\subsection{The \emph{Append} Problem}

We can often mimic value-returning functions with relations
that take an \emph{additional} argument.  For example,  we
can write a function that concatenates \emph{two} lists.

\begin{alltt}
(define concat
  (lambda (xs ys)
    (cond
      [(null? xs) ys]
      [else (cons (car xs) (concat (cdr xs) ys))])))
\end{alltt}

\begin{alltt}
> (concat '(a b c) '(u v))

(a b c u v)
\end{alltt}

\noindent
or the equivalent

\begin{alltt}
> (exists (q) (cdr (solution (fun-call concat q '(a b c) '(u v))))))

((a b c u v) (a b c) (u v))
\end{alltt}

\noindent
And we can write the corresponding relation over \emph{three} lists,

\begin{alltt}
(define concat
  (extend-relation
    (fact (xs) '() xs xs)
    (relation _ (x xs ys zs)
      (to-show `(,x . ,xs) ys `(,x . ,zs))
      (concat xs ys zs))))
\end{alltt}        
\noindent

\begin{alltt}
> (exists (q) (solve 6 (concat '(a b c) '(u v) q)))

(((a b c) (u v) (a b c u v)))
\end{alltt}

\noindent which determines that there is only one answer and which
shows if we concatenate \texttt{(a b c)} to \texttt{(u v)}, we get
\texttt{(a b c u v)}. But, we can move \texttt{q} to another position.

\begin{alltt}
> (exists (q) (solve 6 (concat '(a b c) q '(a b c u v))))

(((a b c) (u v) (a b c u v)))
\end{alltt}

\noindent
This time we determine that \texttt{q} should be \texttt{(u v)}, which
is \emph{not} possible with \texttt{concat} as a function.
Similarly, we can determine that \texttt{q} is \texttt{(a b c)}.

\begin{alltt}
> (exists (q) (solve 6 (concat q '(u v) '(a b c u v))))

(((a b c) (u v) (a b c u v)))
\end{alltt}

\noindent
But what if we include another variable?
\begin{alltt}
> (exists (q r) (solve 6 (concat q r '(a b c u v))))

((() (a b c u v) (a b c u v))
 ((a) (b c u v) (a b c u v))
 ((a b) (c u v) (a b c u v))
 ((a b c) (u v) (a b c u v))
 ((a b c u) (v) (a b c u v))
 ((a b c u v) () (a b c u v)))
\end{alltt}

\noindent
We get all the ways that we might concatenate
two lists to form \texttt{(a b c u v)}.
Now, what if we include yet another variable?

\begin{alltt}
> (exists (q r s) (solve 6 (concat q r s)))

((() xs.0 xs.0)
((x.0) xs.0 (x.0 . xs.0))
((x.0 x.1) xs.0 (x.0 x.1 . xs.0))
((x.0 x.1 x.2) xs.0 (x.0 x.1 x.2 . xs.0))
((x.0 x.1 x.2 x.3) xs.0 (x.0 x.1 x.2 x.3 . xs.0))
((x.0 x.1 x.2 x.3 x.4) xs.0 (x.0 x.1 x.2 x.3 x.4 . xs.0)))
\end{alltt}

\noindent
Here we see that the empty list and any list yield that list.  Then we
get all sorts of constructed lists with the first $N$ elements of the
list chosen as variables of that length.  There is no bound on the
number of answers.  

We can also get an unbounded number of answers with only two variables.

\begin{alltt}
> (exists (q r) (solve 6 (concat q '(u v) `(a b c . ,r))))

(((a b c) (u v) (a b c u v))
((a b c x.0) (u v) (a b c x.0 u v))
((a b c x.0 x.1) (u v) (a b c x.0 x.1 u v))
((a b c x.0 x.1 x.2) (u v) (a b c x.0 x.1 x.2 u v))
((a b c x.0 x.1 x.2 x.3) (u v) (a b c x.0 x.1 x.2 x.3 u v))
((a b c x.0 x.1 x.2 x.3 x.4) (u v) (a b c x.0 x.1 x.2 x.3 x.4 u v)))
\end{alltt}

\noindent
The first answer is the one we expect, where \texttt{q} is
instantiated to \texttt{(a b c)} and \texttt{r} is instantiated to
\texttt{(u v)}.  But, then we discover that \texttt{q} could be a bit
longer.  

And here is an unbounded number of answers with a single variable.

\begin{alltt}
> (exists (q) (solve 6 (concat q '() q)))

((() () ())
((x.0) () (x.0))
((x.0 x.1) () (x.0 x.1))
((x.0 x.1 x.2) () (x.0 x.1 x.2))
((x.0 x.1 x.2 x.3) () (x.0 x.1 x.2 x.3))
((x.0 x.1 x.2 x.3 x.4) () (x.0 x.1 x.2 x.3 x.4)))
\end{alltt}

\noindent
Again, the first answer is the one that we expect, but
the others make sense, too, since no matter what we replace
the variables \texttt{x.}\emph{i} with, we create a
legitimate equation.

A program like \texttt{concat} is what excited the
logic programming world.  It was called \texttt{append}
because of the use of that name in Lisp, but since 
\texttt{concat} is defined globally, it would be wise to
avoid overriding the built-in Scheme function, \texttt{append}.

\subsection{The Type-Inference Problem}

The second famous problem is the type-inference problem.  We start by
considering integers and booleans.  Next, we include some familiar
primitives.  When we are comfortable with those features we include
conditionals, followed by lexical variables, then \texttt{lambda},
application, and \texttt{fix} expressions. Finally, we include
polymorphic \texttt{let}.  Type inference allows for the system to
determine a unique type if the expression has one. 

\subsubsection{Buiding a type inferencer with small relations}

The language for which we infer a type is basically a lambda-calculus
variant of Scheme.  We have chosen, however, to parse this variant
into a language where every expression has a tag as in \texttt{parse},
below.  We have also included an \texttt{unparse}, below, to get back
the original Scheme variant.

\begin{alltt}
(define parse
  (lambda (e)
    (cond
      [(symbol? e) `(var ,e)]
      [(number? e) `(intc ,e)]
      [(boolean? e) `(boolc ,e)]
      [else 
        (case (car e)
          [(zero?) `(zero? ,(parse (cadr e)))]
          [(sub1) `(sub1 ,(parse (cadr e)))]
          [(+) `(+ ,(parse (cadr e)) ,(parse (caddr e)))]
          [(if) `(if ,(parse (cadr e)) ,(parse (caddr e)) ,(parse (cadddr e)))]
          [(fix) `(fix ,(parse (cadr e)))]
          [(lambda) `(lambda ,(cadr e) ,(parse (caddr e)))]
          [(let) `(let ([,(car (car (cadr e))) ,(parse (cadr (car (cadr e))))])
                    ,(parse (caddr e)))]
          [else `(app ,(parse (car e)) ,(parse (cadr e)))])])))
\end{alltt}
\newpage
\begin{alltt}
(define unparse
  (lambda (e)
    (case (car e)
      [(var) (cadr e)]
      [(intc) (cadr e)]
      [(boolc) (cadr e)]
      [(zero?) `(zero? ,(unparse (cadr e)))]
      [(sub1) `(sub1 ,(unparse (cadr e)))]
      [(+) `(+ ,(unparse (cadr e)) ,(unparse (caddr e)))]
      [(if) `(if ,(unparse (cadr e)) ,(unparse (caddr e)) ,(unparse (cadddr e)))]
      [(fix) `(fix ,(unparse (cadr e)))]
      [(lambda) `(lambda (,(car (cadr e))) ,(unparse (caddr e)))]
      [(let) 
       `(let ([,(car (car (cadr e)))
               ,(unparse (cadr (car (cadr e))))])
          ,(unparse (caddr e)))]
      [(app) `(,(unparse (cadr e)) ,(unparse (caddr e)))])))
\end{alltt}

While you are reading the code for the type system, \texttt{!-}, it is
important to keep in mind that although we are presenting a type
inferencing algorithm, it is just a relatively simple logic program.

\texttt{!-} corresponds to the mathematical symbol $\vdash$
(turnstile) and reads ``we can infer.''  That is, from looking at the
relation \texttt{int-rel} and the definition of \texttt{!-}, below, we
can read it as, ``From \texttt{g} \emph{we can infer} that \texttt{x}
is of type \texttt{int} provided that \texttt{x} is an \texttt{intc}.''
For now, we leave \texttt{g} unspecified.

In the expressions below, we use \texttt{int}, \texttt{bool}, and
\texttt{-->} for our type constructors.  We define \texttt{int}
and \texttt{bool} to avoid using lots of quotes.

\begin{alltt}
(define int 'int)
(define bool 'bool)
\end{alltt}

\begin{alltt}
(define int-rel
  (fact (g x) g `(intc ,x) int))

(define !- int-rel)
\end{alltt}

\begin{alltt}
(define g (exists (g) g))

> (let ([result (solution (!- g (parse 17) int))])
    `(!- ,(car result) ,(unparse (cadr result)) ,(caddr result)))
(!- g.0 17 int)

(define ? (exists (?) ?))

> (let ([result (solution (!- g (parse 17) int))])
    `(!- ,(car result) ,(unparse (cadr result)) ,(caddr result)))
(!- g.0 17 int)
\end{alltt}

In the first example, we verify that \texttt{17} is of type
\texttt{int}.  In the second example, the type is unknown, but
whatever is instantiated to the logic variable \texttt{?} is the
type.  In this case, \texttt{?} is instantiated to \texttt{int}.
The existence of \texttt{g.0} in the answers indicates that \texttt{g}
is uninstantiated, so these work for all possible \texttt{g}s.

As a way to abstract the behavior of our testing technology, we
use \texttt{infer-type}, below, which abbreviates some of the
repetition of the first two examples and supports the idea that
the expression might not have a type.

\begin{alltt}
(define-syntax infer-type
  (syntax-rules ()
    [(infer g term type)
     (cond
       [(solution (!- g (parse term) type))
        => (lambda (result)
             `(!- ,(car result) ,(unparse (cadr result)) ,(caddr result)))]
       [else #f])]))
\end{alltt}

Next, we include \texttt{bool-rel}, below, in \texttt{!-}.
\begin{alltt}
(define bool-rel
  (fact (g x) g `(boolc ,x) bool))

(define !- (extend-relation (a1 a2 a3) int-rel bool-rel))
\end{alltt}

\texttt{Exercise 13:} Test \texttt{infer-type} over true and false.

Before we include relations for the arithmetic primitives, we
introduce an idiom, \texttt{(all! (cut) ...}, below, which surrounds
the antecedent operands with \texttt{cut}.  This means that whenever
there is failure, all the antecedents fail.  We do this because our
type inferencer has this deterministic behavior.  There cannot be any
backtracking.  That is, once a decision is made, it cannot be
reconsidered!

Now we can extend \texttt{!-}, below, with the relations
\texttt{zero?-rel}, \texttt{sub1-rel}, and \texttt{+-rel}, which
correspond to the primitves \texttt{zero?}, \texttt{sub1}, and
\texttt{+}, respectively.

\begin{alltt}
(define zero?-rel
  (relation cut (g x)
    (to-show g `(zero? ,x) bool)
    (all! (cut) (!- g x int))))

(define sub1-rel
  (relation cut (g x)
    (to-show g `(sub1 ,x) int)
    (all! (cut) (!- g x int))))

(define +-rel
  (relation cut (g x y)
    (to-show g `(+ ,x ,y) int)
    (all! (cut) (!- g x int) (!- g y int))))
\end{alltt}

\begin{alltt}
(define !- (extend-relation (a1 a2 a3) !- zero?-rel sub1-rel +-rel))
\end{alltt}

\begin{alltt}
> (infer-type g '(zero? 24) ?)

(!- g.0 (zero? 24) bool)
\end{alltt}

\begin{alltt}
> (infer-type g '(zero? (+ 24 50)) ?)

(!- g.0 (zero? (+ 24 50)) bool)
\end{alltt}

The type system can infer that \texttt{(zero? 24)} is of type
\texttt{bool}, because it can infer that \texttt{24} is of type
\texttt{int}.  It can infer that \texttt{(+ 24 50)} is of type
\texttt{int}, so the answer in the second example must be of type
\texttt{bool}.  We can, of course, make more complicated examples
using \texttt{zero?}, \texttt{sub1}, and \texttt{+}, but if they have
a type, it is \texttt{int} or \texttt{bool}.  For example,

\begin{alltt}
> (infer-type g '(zero? (sub1 (+ 18 (+ 24 50)))) ?)

(!- g.0 (zero? (sub1 (+ 18 (+ 24 50)))) bool)
\end{alltt}

Although our parser (and unparser) expects a larger language, at each
stage of defining \texttt{!-}, we are writing a type inferencer for a
larger and larger language.  When we have defined \texttt{!-} for
\texttt{let}, then we have a type inferencer for the full language.
To reiterate, the language starts out very small! It only contains
integers.  Then, as we progress, it gets bigger.  But, the beauty of
type inferencing, is that these little relations grow naturally.  Of
course, we cannot write the relation for \texttt{zero?}  until we have
a relation for numbers and booleans, so there is a natural ordering to
some extent.

In this type system, we must preserve the property that every
well-typed expression has \emph{one} type.  So, what do we do about
conditionals?  Easy.  We require that not only must the test be of
type \texttt{bool}, but the true branch and the false branch must
have the same type.  In a language without variables, \texttt{lambda}
expressions, applications, \texttt{fix} expressions, and \texttt{let}
expressions, that means that they must both be of type \texttt{int}
or they must both be of type \texttt{bool}.  By extending
\texttt{!-}, we can now handle \texttt{if} expressions.

\begin{alltt}
(define if-rel
  (relation cut (g t test conseq alt)
    (to-show g `(if ,test ,conseq ,alt) t)
    (all! (cut) (!- g test bool) (!- g conseq t) (!- g alt t))))
\end{alltt}

\begin{alltt}
(define !- (extend-relation (a1 a2 a3) !- if-rel))
\end{alltt}

\begin{alltt}
> (infer-type g '(if (zero? 24) 3 4) ?)

(!- g.0 (if (zero? 24) 3 4) int)
\end{alltt}

\noindent
Not surprisingly, we discover that the type of the test is \texttt{bool}
and the type of the entire expression is \texttt{int}.

Next, we include lexical variables, which are represented using
symbols.  What is the type of \texttt{(zero? a)}?  If the type of
\texttt{a} is \texttt{int}, then we know that the type of the entire
expression is \texttt{bool}, but if the type of \texttt{a} is
\texttt{bool}, then the expression does not have a type.  How do we
determine the type of \texttt{a}?  We look in \texttt{g}, which is a
type environment that associates lexical (both generic and
non-generic) variables with types.  So far we have ignored \texttt{g},
but now we consider its content using \texttt{non-generic} (a tag)
variables.  We extend \texttt{!-} to include a relation for variables.

\begin{alltt}
(define var-rel
  (relation cut (g v t)
    (to-show g `(var ,v) t)
    (all! (cut) (env g v t))))

(define !- (extend-relation (a1 a2 a3) !- var-rel))
\end{alltt}

\begin{alltt}
(define non-generic-match-env
  (fact (g v t) `(non-generic ,v ,t ,g) v t))

(define non-generic-recursive-env
  (relation cut (g v t w type-w)
    (to-show `(non-generic ,w ,type-w ,g) v t)
    (all! (cut) (unequal? w v) (instantiated g) (env g v t))))

(define env (extend-relation (a1 a2 a3) non-generic-match-env non-generic-recursive-env)))
\end{alltt}

\begin{alltt}
(define unequal?
  (extend-relation
    (relation cut (a)
      (to-show a a)
      (all cut fail))
    (relation _ (a b)
      (to-show a b)
      (fails fail))))
\end{alltt}

\texttt{Exercise 14:} Implement the global antecedent
\texttt{success}, which fails to fail.  Is it the same as \texttt{(all)}?

\begin{alltt}
> (solution (env `(non-generic b int (non-generic a bool ,g)) 'a ?))

((non-generic b int (non-generic a bool g.0)) a bool)
\end{alltt}

\begin{alltt}
> (infer-type `(non-generic a int ,g) '(zero? a) ?)

(!- (non-generic a int g.0) (zero? a) bool)
\end{alltt}

\begin{alltt}
> (infer-type `(non-generic b bool (non-generic a int ,g)) '(zero? a) ?)

(!- (non-generic b bool (non-generic a int g.0)) (zero? a) bool)
\end{alltt}

The first example tests \texttt{env}.  The environment starts out with
\texttt{int} bound to \texttt{b} and \texttt{bool} bound to
\texttt{a}.  The \texttt{non-generic-recursive} relation succeeds,
since we are looking up \texttt{a}, and then the
\texttt{non-generic-match-env} relation succeeds, since we find
\texttt{a}.  In the second answer, we have one item in the type
environment and the \texttt{non-generic-match-env} relation is
followed. In the third example, we have two items in the type
environment, so we take the \texttt{non-generic-recursive} relation, 
then the \texttt{non-generic-match-env} relation succeeds,
since we have stripped off \texttt{b} and \texttt{bool}, leaving
\texttt{a} and its associated type.

Now that we can deal with lexical (non-generic) variables, we can
consider the relation for \texttt{lambda} expressions by extending
\texttt{!-}.

\begin{alltt}
(define lambda-rel
  (relation cut (g v t body type-v)
    (to-show g `(lambda (,v) ,body) `(--> ,type-v ,t))
    (all! (cut) (!- `(non-generic ,v ,type-v ,g) body t))))

(define !- (extend-relation (a1 a2 a3) !- lambda-rel))
\end{alltt}

\begin{alltt}
> (infer-type 
    `(non-generic b bool (non-generic a int ,g))
    '(lambda (x) (+ x 5))
    ?)

(!- (non-generic b bool (non-generic a int g.0))
    (lambda (x) (+ x 5))
    (--> int int))
\end{alltt}

\begin{alltt}
> (infer-type
    `(non-generic b bool (non-generic a int ,g))
    '(lambda (x) (+ x a))
    ?)

(!- (non-generic b bool (non-generic a int g.0))
    (lambda (x) (+ x a))
    (--> int int))
\end{alltt}

\begin{alltt}
> (infer-type g '(lambda (a) (lambda (x) (+ x a))) ?)

(!- g.0 (lambda (a) (lambda (x) (+ x a)))
         (--> int (--> int int)))
\end{alltt}

In the first answer, we see that we have an arrow (\texttt{-->})
type.  The left argument of the arrow type is the type of argument
coming into the function, and the right argument of the arrow type is
the type of the result going out of the function.  So, the inferred
type is a function whose argument is an integer and whose result is an
integer.  The second answer states that the argument is an integer,
but consults the type environment to make sure that the argument going
out is an integer.  In the third example, we forget about the first
environment, because there are no free variables in the expression. We
see that the argument coming in is an integer, but the result is an
arrow type, which takes in an integer and returns an integer.  Close
inspection of the type (directly aligned below the item it is typing)
shows that for each \texttt{lambda} there is an arrow and for each
formal parameter there is a type.  Also, there is a type for the body
of each \texttt{lambda}.  If we think about the type from the inside
out, we see that \texttt{(+ x a)} is an integer only if \texttt{x} and
\texttt{a} are integers.  That determines the type of the inner
\texttt{lambda} and then the type of the outer \texttt{lambda}.  It is
important that we \emph{can} infer the type of \texttt{lambda}
expressions, even though we do not yet have application in our language.
This should be a bit of a surprise.

We come next to application.  In determining the type of an
application, we know that the operator in an application should be
some arrow type.  Furthermore, once we know that type, we know that
the type going out of that type is the same as the type of the entire
application and we know that the operand of the application must be
the type going into that type.  

\begin{alltt}
(define app-rel
  (relation cut (g t rand rator)
    (to-show g `(app ,rator ,rand) t)
    (exists (t-rand)
      (all! (cut) (!- g rator `(--> ,t-rand ,t)) (!- g rand t-rand)))))

(define !- (extend-relation (a1 a2 a3) !- app-rel))
\end{alltt}

\begin{alltt}
> (infer-type g '(lambda (f) (lambda (x) ((f x) x))) ?)

(g.0 (lambda (f)
       (lambda (x) (app (app (var f) (var x)) (var x))))
     (--> 
       (--> type-v.0 (--> type-v.0 t.0))
       (--> type-v.0 t.0)))
\end{alltt}

Here, the type of \texttt{f} is \texttt{(--> type-v.0 (--> type-v.0 t.0))}, so
the type of \texttt{x} must be \texttt{type-v.0}, and the type of
\texttt{(lambda (x) ((f x) x))} must be \texttt{(--> type-v.0 t.0)}.
As should be evident, once we add a relation for application, things
start to get a bit tricky.  We can no longer rely on aligning the
\texttt{lambda}s with the arrows.  Here we have two \texttt{lambda}s
and four arrows.  Yet another surprise.  Our inferencer is starting
to be clever.

We may be tempted to use our language to write (and test) recursive
functions.  To test the expression, we use the call-by-value \texttt{fix}
primitive:

\begin{alltt}
(define fix-rel
  (relation cut (g rand t)
    (to-show g `(fix ,rand) t)
    (all! (cut) (!- g rand `(--> ,t ,t)))))

(define !- (extend-relation (a1 a2 a3) !- fix-rel))
\end{alltt}

In Scheme, we define \texttt{fix}, below.  Although \texttt{fix} can
be defined using simple \texttt{lambda} terms as in the \texttt{Y}
combinator, \emph{this} type system cannot determine a type for it.
Thus, \texttt{fix} must be primitive and the associated primitive
call's type can be inferred as above.

\begin{alltt}
(define fix
  (lambda (e)
    (e (lambda (z) ((fix e) z)))))
\end{alltt}

\begin{alltt}
> (infer-type
    g
    '((fix (lambda (sum)
             (lambda (n)
               (if (zero? n)
                 0
                 (+ n (sum (sub1 n)))))))
      10)
    ?)

(!- g.0
    ((fix (lambda (sum)
            (lambda (n)
              (if (zero? n)
                0 
                (+ n (sum (sub1 n)))))))
     10)
    int)
\end{alltt}

Let's consider the following expression
\begin{alltt}
> ((fix (lambda (sum)
          (lambda (n)
            (+ n (sum (sub1 n))))))
   10)
\end{alltt}

It fails to terminate.  But, can we infer its type?  Yes, \emph{an
expression may have a type, even if evaluating it would lead to
nontermination.}  This is a confusing aspect of type inference.  We
know from the ``Halting Problem'' that we cannot tell in advance
whether evaluating an expression will terminate, but \emph{this}
type inferencing system is guaranteed to terminate.  Thus, we
can infer the type before we run it.  As a result, information that
the run-time system can learn from the type (or the process of
inferring the type) can be put to good use.  Here is its type.
\newpage
\begin{alltt}
> (infer-type
    g
    '((fix (lambda (sum)
             (lambda (n)
               (+ n (sum (sub1 n))))))
      10)
    ?)

(!- g.0
    ((fix (lambda (sum) (lambda (n) (+ n (sum (sub1 n)))))) 10)
    int)
\end{alltt}

\subsection{Polymorphic \texttt{let}}

The \texttt{let}-expression is a bit more subtle.  Let's take a look at
an expression that should type check, but won't in the absence of
\texttt{let}.

\begin{alltt}
> (infer-type
    g
    '((lambda (f)
        (if (f (zero? 5))
          (+ (f 4) 8)
          (+ (f 3) 7)))
      (lambda (x) x))
    ?)
#f
\end{alltt}

Because \texttt{f} becomes the non-generic identity, once a type for
\texttt{f} is determined, it must stay the same.  Obviously, we would
expect that the evaluation of ``\texttt{((lambda (f) ...) ...)}
to be 10, but it has no type.  If, however, we change the
expression to use \texttt{let}
\begin{alltt}
(let ([f (lambda (x) x)])
  (if (f (zero? 5))
    (+ (f 4) 8)
    (+ (f 3) 7)))
\end{alltt}

\noindent
and think about $\beta$-substituting for \texttt{f} throughout, then
we can see that this expression should have a type, \texttt{int}.
Instead of doing the substitution, we mark certain variables
\emph{generic} as they are placed in the environment. 

This is the \emph{polymorphic} \texttt{let}, since the variable is
tagged with \texttt{generic} in the environment.  If the variable were
tagged with \texttt{non-generic}, then this would be the familiar
\texttt{let}.  We have only to determine what happens in environment
lookup when a variable with a \texttt{generic} tag is an arrow type.
Those who wish to include a more general \texttt{let} expression, one
whose binding variable is bound to a \texttt{non-generic}, feel free
to do so, but for our purposes, we assume that all right-hand sides of
\texttt{let} expressions are \texttt{lamdda} expressions.

\begin{alltt}
(define polylet-rel
  (relation cut (g v rand body t)
    (to-show g `(let ([,v ,rand]) ,body) t)
    (exists (t-rand)
      (all! (cut)
        (!- g rand t-rand)
        (!- `(generic ,v ,t-rand ,g) body t)))))

(define !- (extend-relation (a1 a2 a3) !- polylet-rel))
\end{alltt}

In order to implement these generics, we introduce a relation,
\texttt{instantiate}, which uses \texttt{fun-nocheck}, whose purpose
is to associate the type \texttt{(--> targ tresult)} with the type
\texttt{t}.

\begin{alltt}
(define instantiate
  (letrec
    ([instantiate-term
       (lambda (t env)
         (cond
           [(var? t)
            (cond
              [(assv t env)
               => (lambda (pr)
                    (values (cdr pr) env))]
              [else (let ([new-var (var (var-name t))])
                      (values new-var (cons `(,t . ,new-var) env)))])]
           [(pair? t)
            (let-values (a-t env) (instantiate-term (car t) env)
              (let-values (d-t env) (instantiate-term (cdr t) env)
                (values (cons a-t d-t) env)))]
           [else (values t env)]))])
    (lambda (t)
      (let-values (ct env) (instantiate-term t '())
        ct))))
\end{alltt}
\newpage
\begin{alltt}
(define generic-base-env
  (relation cut (g v targ tresult t)
    (to-show `(generic ,v (--> ,targ ,tresult) ,g) v t)
    (all! (cut) (instantiate t `(--> ,targ ,tresult)))))

(define generic-recursive-env
  (relation cut (g v w type-w t)
    (to-show `(generic ,w ,type-w ,g) v t)
    (all! (cut) (env g v t))))

(define generic-env
  (extend-relation (a1 a2 a3) generic-base-env generic-recursive-env))

(define env (extend-relation (a1 a2 a3) env generic-env))
\end{alltt}

Now that we have extended our environments to handle generic
as well as non-generic variables, we can infer the right type.

\begin{alltt}
> (infer-type 
    g
    '(let ([f (lambda (x) x)])
       (if (f (zero? 5))
         (+ (f 4) 8)
         (+ (f 3) 7)))
    ?)

(!- g.0
    (let ([f (lambda (x) x)])
      (if (f (zero? 5))
        (+ (f 4) 8)
        (+ (f 3) 7)))
    int)
\end{alltt}

\subsubsection{Long cuts}
Each call returns whenever the computation backs into the cut, but
that means that recursive calls just return by following the control
stack of procedure calls.  Instead, we would prefer to pick the point
where we want to be whenever failure backs into a cut.  This is
possible and we demonstrate it by passing a specific cut into the
function \texttt{!-generator}, below.  Thus no matter how deeply
nested we are in an expression that we wish to type check, as soon as
an invalid expression occurs or we have a type violation, we jump
\emph{way} out.  In \texttt{all!}, it takes the
the long \texttt{cut}, not a local short one.

\newpage
\begin{alltt}
(define !-generator
  (lambda (long-cut)
    (letrec
      ([!- (extend-relation (a1 a2 a3)
             (relation _ (g v t)
               (to-show g `(var ,v) t)
               (all! (long-cut) (env g v t)))
             (relation _ (g x)
               (to-show g `(intc ,x) int)
               (all! (long-cut)))
             (relation _ (g x)
               (to-show g `(boolc ,x) bool)
               (all! (long-cut)))
             (relation _ (g x)
               (to-show g `(zero? ,x) bool)
               (all! (long-cut) (!- g x int)))
             (relation _ (g x)
               (to-show g `(sub1 ,x) int)
               (all! (long-cut) (!- g x int)))
             (relation _ (g x y)
               (to-show g `(+ ,x ,y) int)
               (all! (long-cut) (!- g x int) (!- g y int)))
             (relation _ (g t test conseq alt)
               (to-show g `(if ,test ,conseq ,alt) t)
               (all! (long-cut) (!- g test bool) (!- g conseq t) (!- g alt t)))
             (relation _ (g v t body type-v)
               (to-show g `(lambda (,v) ,body) `(--> ,type-v ,t))
               (all! (long-cut) (!- `(non-generic ,v ,type-v ,g) body t)))
             (relation _ (g t rand rator)
               (to-show g `(app ,rator ,rand) t)
               (exists (t-rand)
                 (all! (long-cut)
                   (!- g rator `(--> ,t-rand ,t))
                   (!- g rand t-rand))))
             (relation _ (g rand t)
               (to-show g `(fix ,rand) t)
               (all! (long-cut) (!- g rand `(--> ,t ,t))))
             (relation _ (g v rand body t)
               (to-show g `(let ([,v ,rand]) ,body) t)
               (exists (t-rand)
                 (all! (long-cut)
                   (!- g rand t-rand)
                   (!- `(generic ,v ,t-rand ,g) body t)))))])
      !-)))
\end{alltt}
\newpage
\begin{alltt}
(define !-
  (relation cut (g exp t)
    (to-show g exp t)
    ((!-generator cut) g exp t)))
\end{alltt}

\noindent
And, now we can test it with \texttt{infer-type} as before.

\subsubsection{Type inhabitation}

We are going to do an experiment and in order to get the results
we want, we need to respecify the order of the relations.  Thus
we redefine \texttt{!-}.

\begin{alltt}
(define !-
  (extend-relation (a1 a2 a3)
    var-rel int-rel bool-rel zero?-rel sub1-rel +-rel 
    if-rel lambda-rel app-rel fix-rel polylet-rel))
\end{alltt}

Here are four, perhaps unexpected, examples.

\begin{alltt}
> (let ([res (exists (g ?)
               (solution (!- g ? '(--> int int))))])
    `(!- ,(car res) ,(unparse (cadr res)) ,(caddr res)))

(!- (non-generic v.0 (--> int int) g.0)
    v.0
    (--> int int))
\end{alltt}

\begin{alltt}
> (let ([res (exists (g la f b)
               (solution
                 (!- g `(,la (,f) ,b) '(--> int int))))])
    `(!- ,(car res) ,(unparse (cadr res)) ,(caddr res)))

(!- g.0 (lambda (v.0) v.0) (--> int int))
\end{alltt}

\begin{alltt}
> (let ([res (exists (g h r q z y t)
               (solution 
                  (!- g `(,h ,r (,q ,z ,y)) t)))])
    `(!- ,(car res) ,(unparse (cadr res)) ,(caddr res)))

(!- (non-generic v.0 int g.0) (+ v.0 (+ v.0 v.0)) int)
\end{alltt}

\begin{alltt}
> (let ([res (exists (g h r q z y t u v)
               (solution
                 (!- g `(,h ,r (,q ,z ,y)) `(,t ,u ,v))))])
    `(!- ,(car res) ,(unparse (cadr res)) ,(caddr res)))

(!- g.0 (lambda (v.0) (+ v.0 v.0)) (--> int int))
\end{alltt}

The first example attempts to find an expression whose type is
\texttt{(--> int int)}, but instead finds a type environment that
binds that type to the variable \texttt{v.0}, and then the expression is
trivially \texttt{v.0}.  The second example produces an expression
given the type.  This is answering the question, ``What expression
\emph{inhabits} that type?''  In our case, the identity function
inhabits that type.  But, to make these first two examples work, we
had to place \texttt{var-rel} first in the definition of
\texttt{!-}, above.  In the third example, it infers that \texttt{t}
must be of \texttt{int} type.  Then since there is only one binary
operation that returns an \texttt{int} (i.e., \texttt{+}), it
determines \texttt{q} and \texttt{h}.  Next, we can infer that \texttt{r},
\texttt{z}, and \texttt{y} must be of \texttt{int} type, and what is
easier than making them all the same variable and placing it in the
initial type environment.  The last example only differs in the shape
of the resultant type.  Here it assumes that since the type contains
three parts, it must be an arrow type.  That means that \texttt{h}
must be the symbol \texttt{lambda}. Once again the only binary
operator is \texttt{+} making \texttt{z} and \texttt{y} be of
\texttt{int} type.

\subsection{Prolog's \texttt{name} as a relation}

Consider treating invertible binary operators as three-place
relations. The function
\texttt{invertible-binary-function->ternary-relation}, below, expects
that at most one of the three arguments is a variable and solves the
problem by determining which of the three variables is uninstantiated.

\begin{alltt}
(define invertible-binary-function->ternary-relation
  (lambda (op inverted-op)
    (extend-relation (a1 a2 a3)
      (relation _ (x y z)
        (to-show x y z)
        (all (fails (instantiated z)) (fun-call op z x y)))
      (relation _ (x y z)
        (to-show x y z)
        (all (fails (instantiated y)) (fun-call inverted-op y z x)))
      (relation _ (x y z)
        (to-show x y z)
        (all (fails (instantiated x)) (fun-call inverted-op x z y)))
      (relation _ (x y z)
        (to-show x y z)
        (fun-call op z x y)))))

(define ++ (invertible-binary-function->ternary-relation + -))
(define -- (invertible-binary-function->ternary-relation - +))
(define ** (invertible-binary-function->ternary-relation * /))
(define // (invertible-binary-function->ternary-relation / *))

> (exists (x) (solution (++ x 16.0 8)))
(-8.0 16.0 8)

> (exists (x) (solution (** 10 x 50)))

(10 5 50)

> (exists (x) (solution (-- 10 7 x)))

(10 7 3)
\end{alltt}

And we can do something similar with invertible unary functions.
\begin{alltt}
(define invertible-unary-function->binary-relation
  (lambda (op inverted-op)
    (extend-relation (a1 a2)
      (relation _ (x y)
        (to-show x y)
        (all (fails (instantiated y)) (fun-call op y x)))
      (relation _ (x y)
        (to-show x y)
        (all (fails (instantiated x)) (fun-call inverted-op x y)))
      (trace-relation third _ (x y)
        (to-show x y)
        (begin (pretty-print "Third rule") (fun-call op y x))))))
\end{alltt}

\begin{alltt}
(define symbol->lnum
  (lambda (sym)
    (map char->integer (string->list (symbol->string sym)))))

(define lnum->symbol
  (lambda (lnums)
    (string->symbol (list->string (map integer->char lnums)))))

(define name
  (invertible-unary-function->binary-relation symbol->lnum lnum->symbol))
\end{alltt}
\begin{alltt}
> (exists (x) (solution (name 'sleep x)))

(sleep (115 108 101 101 112))

> (exists (x) (solution (name x '(115 108 101 101 112))))

(sleep (115 108 101 101 112))
\end{alltt}

In the first example, we return the \texttt{char->integer} of each
character in \texttt{sleep}.  In the second, given a list of integers,
presumably derived from \texttt{char->integer}, it returns the symbol
made from those integers.  Thus, we have the Prolog relation
\texttt{name}.  For our purposes, which is an embedding in Scheme, it
is probably unnecessary, but it is interesting, nonetheless, that we
can define these two relation-generating Scheme functions.  There are
more unary functions that can be so treated, like
\texttt{symbol->string} and \texttt{string->symbol}, but we leave
their inclusion to the programmer who might need them.

\section{Final thoughts}

In a relation call such as \texttt{(exists (x) (foo x 'a y))} we have
three different kinds of values.  The symbol \texttt{'a} is a raw
value, the lexical variables, \texttt{foo} and \texttt{y}, become
values as part of the call, and the logic variable \texttt{x} becomes
a value (or is shared with another logic variable) when its argument
unifies with another value.  Having the call allow for their
intermingling clarifies why we need to manage their scopes.  Because
the call is awaiting other arguments, we do not need to think about
this in terms of a finished call that returns a value until these
additional arguments are absorbed by the result of the relation call.

We have given up two main features of logic programming?  First, we do
not have meta-level operations that allow for construction of relations
from existing relations.  This may be retrieved by building the relations
as data structures and using \texttt{eval} to construct the actual
relations when they are needed.  This solution leaves much to be
desired and the way Prolog handles this is cleaner, especially since
\texttt{eval} should only be used in the rarest of circumstances.  It
may be possible, however, with higher-order capabilities to get
around most of these needs.  Second, and more importantly, we have
abandoned the database capability associated with logic
programming. By that, we mean the ability for relations to be
treated as a global monolithic relation.  To circumvent this
shortcoming, one can write a driver that knows some subset (possibly
all) of the relations and their arities and whenever a relation of a
particular arity is invoked, it searches through all the relations of
that arity.  This would be easy to set up with
\texttt{extend-relation} and an association list that associates an
arity with an extended relation of that arity. We can further
partition this association list by replacing the arity with a type
signature.  But, this approach would have to re-address the cut.
Since, this browsing capability is not required by the problem domains
we have in mind, probably it is best that it be developed on an as
needed basis.  From the start we have been developing a tool that
would allow for easy implementation of language-related programs such
as type inferencers, interpreters, and compilers.  These clearly do
not need a global database.

One disadvantage of this implementation is that there is no obvious
place to change the depth-first search strategy to one that supports
breadth-first search.  In its place we have made relations
lexically-scoped, first-class, and extensible, which seems to be a
fair tradeoff.  One approach that might work for getting back other
search strategies works like this.  The formals to \texttt{lambda@}
could be re-ordered to have the last two always be \texttt{cutk} and
\texttt{fk}.  This would mean that there could be just one argument
instead of four.  That one argument would be a stream of say,
\emph{package}s, where a package contains everything but \texttt{fk}.
Then, each package, which now contains \texttt{sk} \texttt{subst}, and
\texttt{cutk} could, in turn, be treated as a stream of \emph{pack}s,
where a pack contains everything but \texttt{cutk}.  A pack could be
implemented with \texttt{cons}.  Now that these four variables can be
treated like a stream of streams, we can think about different ways of
combining streams to perhaps yield different and interesting search
strategies.  But, this is left as an exercise for the reader.  Be wary
that not every use of \texttt{@} physically takes four arguments, so
there is a little bookkeeping to get this to work.  The main thing to
do is change the \texttt{lambda@} into \texttt{lambda} first.  Once
that is running, changing the argument order and turning the data into
streams is simple.

Our goal has been to present the ideas of logic programming without
using a lot of special features of Scheme and without losing the feel
of programming in Scheme.  Now, we can say that the basic unsullied
logic system interface contains five operators: \texttt{relation},
\texttt{extend-relation}, \texttt{all}, \texttt{all!}, and
\texttt{solve}.
\section{Acknowledgments}

This paper would not have been possible without the earlier work on
implementing logic systems with Anurag Mendhekar.  The work with
Anurag led to Jon Rossie's use of our logic system in the development
of his dissertation's results.  His utilization of the tool helped us
understand how we had to weave things together.  But, it wasn't until
work with Mitch Wand and Chris Haynes in \emph{Essentials of
Programming Languages, Second Edition} on the material on unification,
substitutions, and logic programming that it seemed there might be a
chance for a ``Poor Man's'' solution.  Steve Ganz's dissertation also
needed an inference system.  Over the years, Steve began to see how we
could make the seamlessness of Scheme possible by showing how to
partition the monolithic relations into functions.  Such a function
then represents a relation that can be invoked as a function.  Next we
noticed that the sullied operators could also be written as functions
with the same interface, thus removing a dispatch.  Once the dispatch
was removed, it was easy to remove the lone remaining search down the
antecedents.  Discovering that a non-recursive unifier was possible
came much later.  The first implementation and discussion of
polymorphic \texttt{let} is the work of Jeremiah Willcock.  Treating
the consequent of a relation as an antecedent has been inspired by
Seres and Spivey's paper.  Several fruitful discussions with Venkatesh
Choppella led to more thought about the types in the code of this
logic system.  We are grateful to Oscar Waddell for implementing
\texttt{expand-only}, the partial macro expander.  Regretably, it
cannot be made fully general, however, it does meet our needs.
\end{document}
\bibliography{rollyourown}
\end{document}

